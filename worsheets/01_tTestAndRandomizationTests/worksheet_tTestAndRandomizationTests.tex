\documentclass[12pt]{article}

\usepackage[shortlabels]{enumitem}
\usepackage{hyperref}
\usepackage[margin=1cm]{geometry}

\title{Worksheet: hypothesis tests}
\author{Joaquin Rapela}

\begin{document}

\maketitle

\begin{enumerate}

    \item Perform a detailed hypothesis test for example 2 in the discussion
        notes.

	\item A random sample of $n=35$ observations from a quantitative population
produced a mean $\bar{x}=2.4$ and a standard deviation $s=0.29$. Suppose your
research objective is to show that the population mean $\mu$ exceeds 2.3.

		\begin{enumerate}[(a)]
			\item Give the null and alternative hypotheses for the test.
			\item Locate the rejection region for the test using a 5\% significance level.
			% \item Find the standard error of the mean.
			\item Before you conduct the test, use your intuition to decide
whether the sample mean $\bar{x}=2.4$ is likely or unlikely, assuming that
$\mu=2.3$. Now conduct the test. Do the data provide sufficient evidence to
indicate that $\mu>2.3$? 
		\end{enumerate}

	\item \textbf{Potency of an Antibiotic} A drug manufacturer claimed that the mean
	potency of one of its antibiotics was 80\%. A random sample of $n=100$
	capsules were tested and produced a sample mean of $\bar{x}=79.7\%$ with a
	standard deviation of $s=0.8\%$. Do the data present sufficient evidence to
	refute the manufacturer’s claim? Let $\alpha=.05$.

		\begin{enumerate}[(a)]
			\item State the null hypothesis to be tested.
			\item State the alternative hypothesis.
			\item Conduct a statistical test of the null hypothesis and state your conclusion.
		\end{enumerate}

	\item \textbf{Smoking and Lung Capacity} It is recognized that cigarette smoking has a deleterious effect on lung function. In a study of the effect of cigarette smoking on the carbon monoxide diffusing capacity (DL) of the lung, researchers found that current smokers had DL readings significantly lower than those of either exsmokers or nonsmokers. The carbon monoxide diffusing capacities for a random sample of $n=20$ current smokers are listed here:

	103.768 92.295 100.615 102.754 88.602 61.675 88.017 108.579 73.003 90.677 71.210 73.154 123.086 84.023 82.115 106.755 91.052 76.014 89.222 90.479

	Do these data indicate that the mean DL reading for current smokers is significantly lower than 100 DL, the average for nonsmokers? Use a $\alpha=.01$.

    \item (optional) solve problems from Prof. Harris' worksheet on
        \href{https://drive.google.com/file/d/1nzu23HT-\_4eidWs6O29J68PxX3-eUTzI/view}{t-test
        and randomization test} and/or on
        \href{https://drive.google.com/file/d/1csFdruHPSSH0c\_JBEWhZyR0dQ2dmlyS6/view}{making
        a raster plot}.

\end{enumerate}

\end{document}
