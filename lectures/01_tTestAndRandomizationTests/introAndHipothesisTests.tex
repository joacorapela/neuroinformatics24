
\documentclass{beamer}

\mode<presentation> {

% The Beamer class comes with a number of default slide themes
% which change the colors and layouts of slides. Below this is a list
% of all the themes, uncomment each in turn to see what they look like.

%\usetheme{default}
%\usetheme{AnnArbor}
%\usetheme{Antibes}
%\usetheme{Bergen}
%\usetheme{Berkeley}
%\usetheme{Berlin}
%\usetheme{Boadilla}
%\usetheme{CambridgeUS}
%\usetheme{Copenhagen}
%\usetheme{Darmstadt}
%\usetheme{Dresden}
%\usetheme{Frankfurt}
%\usetheme{Goettingen}
%\usetheme{Hannover}
%\usetheme{Ilmenau}
%\usetheme{JuanLesPins}
%\usetheme{Luebeck}
%\usetheme{Madrid}
%\usetheme{Malmoe}
%\usetheme{Marburg}
%\usetheme{Montpellier}
\usetheme{PaloAlto}
%\usetheme{Pittsburgh}
%\usetheme{Rochester}
%\usetheme{Singapore}
%\usetheme{Szeged}
%\usetheme{Warsaw}

% As well as themes, the Beamer class has a number of color themes
% for any slide theme. Uncomment each of these in turn to see how it
% changes the colors of your current slide theme.

%\usecolortheme{albatross}
%\usecolortheme{beaver}
%\usecolortheme{beetle}
%\usecolortheme{crane}
%\usecolortheme{dolphin}
%\usecolortheme{dove}
%\usecolortheme{fly}
%\usecolortheme{lily}
%\usecolortheme{orchid}
%\usecolortheme{rose}
%\usecolortheme{seagull}
%\usecolortheme{seahorse}
%\usecolortheme{whale}
%\usecolortheme{wolverine}

%\setbeamertemplate{footline} % To remove the footer line in all slides uncomment this line
%\setbeamertemplate{footline}[page number] % To replace the footer line in all slides with a simple slide count uncomment this line

%\setbeamertemplate{navigation symbols}{} % To remove the navigation symbols from the bottom of all slides uncomment this line
}

% remove title and author from left panel
 \makeatletter
  \setbeamertemplate{sidebar \beamer@sidebarside}%{sidebar theme}
  {
    \beamer@tempdim=\beamer@sidebarwidth%
    \advance\beamer@tempdim by -6pt%
    \insertverticalnavigation{\beamer@sidebarwidth}%
    \vfill
    \ifx\beamer@sidebarside\beamer@lefttext%
    \else%
      \usebeamercolor{normal text}%
      \llap{\usebeamertemplate***{navigation symbols}\hskip0.1cm}%
      \vskip2pt%
    \fi%
  }%
\makeatother
% done remove title and author from left panel 

\hypersetup{colorlinks,citecolor=}
\usepackage{graphicx} % Allows including images
\usepackage{booktabs} % Allows the use of \toprule, \midrule and \bottomrule in tables
\usepackage{natbib}
\usepackage{apalike}
\usepackage{comment}
% \usepackage{enumitem}
% \setlist[itemize]{topsep=0pt,before=\leavevmode\vspace{-1.5em}}
% \setlist[description]{style=nextline}
\usepackage{amsthm}
\usepackage{media9}
% \usepackage{multimedia}
\usepackage{hyperref}
\usepackage{tikz}
\tikzset{
     arrow/.style={-{Stealth[]}}
     }
\usetikzlibrary{positioning,arrows.meta}
\usetikzlibrary{shapes.geometric}

\setbeamertemplate{navigation symbols}{}%remove navigation symbols

\usepackage{setspace}

% \newtheorem{example}{Example}
% \setbeamertemplate{theorems}[numbered]

\newcounter{saveenumi}
\newcommand{\seti}{\setcounter{saveenumi}{\value{enumi}}}
\newcommand{\conti}{\setcounter{enumi}{\value{saveenumi}}}
\newcommand{\keepi}{\addtocounter{saveenumi}{-1}\setcounter{enumi}{\value{saveenumi}}}

%----------------------------------------------------------------------------------------
%	TITLE PAGE
%----------------------------------------------------------------------------------------

\title{Foundations of probability theory}

\author{Joaqu\'{i}n Rapela} % Your name
\institute[GCNU, UCL] % Your institution as it will appear on the bottom of every slide, may be shorthand to save space
{
Gatsby Computational Neuroscience Unit\\University College London % Your institution for the title page
}
\date{\today} % Date, can be changed to a custom date

\AtBeginSection[]
  {
     \begin{frame}<beamer>
     \frametitle{Contents}
         \tableofcontents[currentsection,hideallsubsections]
     \end{frame}
  }

\begin{document}

\begin{frame}
\titlepage % Print the title page as the first slide
\end{frame}

\begin{frame}
\frametitle{Contents} % Table of contents slide, comment this block out to remove it
\tableofcontents % Throughout your presentation, if you choose to use \section{} and \subsection{} commands, these will automatically be printed on this slide as an overview of your presentation
\end{frame}

\section{Course logistics}

\begin{frame}
\frametitle{Course logistics}
\end{frame}

\section{Statistical remarks}

\begin{frame}
    \frametitle{}

    \begin{example}

        Is the average height of male from New Zeland larger than six feet?

    \end{example}

\end{frame}

\begin{frame}
\frametitle{Statistical remarks}

    \begin{enumerate}

        \item It is frequent in statistics to assume that observed data follows
            a given probability distribution. For example a:

            \begin{itemize}

                \item normal distribution with parameters mean $\mu$ and variance
                    $\sigma^2$, $\mathcal{N}(\mu,\sigma^2)$,

                \item Poisson distribution with expected rate parameter $\lambda$,
                    $\mathcal{P}(\lambda)$,

                \item Binomial distribution with number of observation parameter $n$
                    and with a success probability parameter $p$,
                    $\mathcal{B}(n,p)$.

            \end{itemize}

            \onslide<2-> {
            \begin{example}

                Assume that the height of males from New Zeland follows of
                normal distribution with mean $\mu$ and variance
                $\sigma^2=0.25$ (i.e., $h\sim\mathcal{N}(\mu,0.25$).

            \end{example}
            }
            \seti
    \end{enumerate}

\end{frame}

\begin{frame}
\frametitle{Statistical remarks}

    \begin{enumerate}

        \conti

        \item One branch of statistics, \textbf{estimation theory}, provides
            tools to estimate these parameters from observations.

        \item Another branch of statistics, \textbf{hypothesis testing},
            provides tools to make statistically-informed decisions about the
            value of these parameters.

            \seti

    \end{enumerate}

\end{frame}

\begin{frame}
\frametitle{Statistical remarks}

    \begin{enumerate}[<+->]

        \conti

        \item To estimate parameters, or to make decisions about them, we use
            \textbf{observations}, $x_1,\ldots,x_N$.

            \begin{example}

                We will measure the height of a sample of $N=10$ males from New
                Zeland, $h_1,\ldots,h_N$.

                To avoid flying to New Zeland, we will simulate
                $h_i,\ldots,h_N$ as random samples from a normal distribution
                with mean $\mu=6\,\text{feet}$ and variance
                $\sigma^2=0.25\,\text{feet}^2$ .

            \end{example}

            \seti

    \end{enumerate}

\end{frame}

\begin{frame}
\frametitle{Statistical remarks}

    \begin{enumerate}

        \conti

        \item A goal of statistics is to \textbf{infer properties of the
            population} (e.g., the mean height of all males in New Zeland) from
            \textbf{properties of the sample} (e.g., the heights of the sample
            of 10 individuals).

    \end{enumerate}

\end{frame}

\section{Hypothesis testing}

\begin{frame}
\frametitle{Null and alternative hypothesis}

    \begin{itemize}

        \item In hypothesis testing we work with a \textbf{null hypothesis},
            $\mathcal{H}_0$, and an \textbf{alternative hypothesis},
            $\mathcal{H}_a$, collect a sample of data $x_1,\ldots,x_N$, and
            test if this data provides sufficient statistical evidence in favor
            of the alternative hypothesis. If this happens we reject the null
            hypothesis.

        \item However, if the collected data does not provide sufficient
            statistical evicence in favor of the aternative hypothesis, we do
            not accept the null hypothesis, but we say that we failed to reject
            it.  \textbf{Hypothesis tests do not prove null hypothesis, they
            only provide statistical evidence to reject it, or fail to reject it.}

    \end{itemize}

\end{frame}

\begin{frame}
\frametitle{Null and alternative hypothesis}

    \begin{example}

        Is the average height of males from New Zeland larger than six feet?

        \begin{description}

            \item[$\mathcal{H}_0$]: the average height of males from New Zeland
                is six feet (i.e., $\mu=6\,\text{feet}$).

            \item[$\mathcal{H}_a$]: the average height of males from New Zeland
                is larger than six feet (i.e., $\mu>6\,\text{feet}$).

        \end{description}

    \end{example}

\end{frame}

\begin{frame}
\frametitle{Test statistic and its sampling distribution}

    \begin{itemize}

        \item To perform a hypothesis test we propose a \textbf{test statistic}, a
    function of the sample data, like the sample mean:

            \begin{align}
                \bar{x}=\frac{1}{N}\sum_{i=1}^Nx_i\label{eq:sampleMean}
            \end{align}

        \item Because the sample data is random, the test statistic is also
            random. To perform hypothesis tests we need to know the
            distribution of the test statistic, which is called the
            \textbf{sampling distribution}.

        \item If the observed test statistic is highly unprobable under
            $\mathcal{H}_0$ (based on the sampling distribution), we reject the
            null hypothesis.

    \end{itemize}

\end{frame}

\begin{frame}
\frametitle{Test statistic and its sampling distribution}

    \begin{example}
        \begin{itemize}

            \item to summarize the information in the sample of 10 heights we
                will use the sample mean of heights, $\bar{h}$,
                Eq.~\ref{eq:sampleMean}.

            \item from probability theory we know that, if
                $x_i\sim\mathcal{N}(\mu,\sigma^2)$, then the mean of the
                independent random variables $x_1,\ldots,x_N$ is
                $\bar{x}\sim\mathcal{N}(\mu, \frac{\sigma^2}{N})$. Thus, under
                $\mathcal{H}_0$ $\bar{h}\sim\mathcal{N}(6, \frac{0.25}{10})$.

        \end{itemize}
    \end{example}

\end{frame}

\begin{frame}
\frametitle{Rejection and non-rejection regions}

    Having calculated the sampling distribution under the null hypothesis, we
    partition the space of all possible values that the test statistic can take
    into a reject and a non-reject region. The \textbf{reject region} is a
    region of low probability under the null hypohesis, which is consitent with
    alternative hypothesis. The \textbf{non-reject region}, is a region of
    large probability under the null hypothesis, that is inconsistent with the
    alternative hypothesis.

    \begin{center}
        \includegraphics[width=3in]{figures/lectureEx_pdf_xbar.png.png}
    \end{center}
\end{frame}

\begin{frame}
\frametitle{Two types of errors, confidence leval and p-value}

    \begin{description}

        \item[Type I error]: reject the null hypothesis when it is true.

        \item[Type II error]: not reject the null hypothesis when the
            alternative one is true.

    \end{description}

    Some statistical tests are designed to constrain the probability of type I
    error, called the \emph{confidence level, $\alpha$} of the test. Most
    commonly $\alpha=0.05$.

    Other tests compute the observed value of the test statistic,
    $t_\text{obs}$ and calculate the probability that the test statistic is
    larger or equal than its observed value.  This probability is called the
    \textbf{p-value} of the test.

\end{frame}

\begin{frame}
\frametitle{Steps to perform a hypothesis test}

    We start assuming that we know the distribution of the experimental samples
    and we agree on a sample size $N$ (in
    Section~\ref{sec:selectionOfSamleSize} we discuss how to select and optimal
    sample size). We set the confidence level $\alpha$ of the test (i.e., the
    probability that the test statistic falls in the reject region given that
    the null hypothesis is valid).

    \begin{enumerate}

        \item collect an experimental sample $x_1,\ldots,x_N$.

        \item compute the value of the test statistic corresponding to the
    collected experimental sample (e.g., sample mean, Eq.~\ref{eq:sampleMean})

        \item calculate the sample statistic, $t_\text{obs}$.

        \item for testing based on confidence level:

            \begin{enumerate}[a]

                \item divide the space of all possible values of the test statistic
            on a reject and a non-reject regons at a confidence level $\alpha$.

                \item if the value of the test statistic falls in the reject
            region, reject the null hypothesis at a confidence $\alpha$. If it
            does not, do not reject the null hypothesis at this confidence.

            \end{enumerate}

        \item for testing based on p-value:

            \begin{enumerate}[a]

                \item calculate the p-value (i.e., the probability that the test
            statistic is larger than the observed one.

                \item reject the null hypothesis if the p-value is lower than the
            agreed confidence level $\alpha$, and do not reject it otherwise.

            \end{enumerate}

    \end{enumerate}

\end{frame}

\begin{frame}
\frametitle{Central limit theorem}

\end{frame}

\end{document}
