
\documentclass{beamer}

\mode<presentation> {

% The Beamer class comes with a number of default slide themes
% which change the colors and layouts of slides. Below this is a list
% of all the themes, uncomment each in turn to see what they look like.

%\usetheme{default}
%\usetheme{AnnArbor}
%\usetheme{Antibes}
%\usetheme{Bergen}
%\usetheme{Berkeley}
%\usetheme{Berlin}
%\usetheme{Boadilla}
%\usetheme{CambridgeUS}
%\usetheme{Copenhagen}
%\usetheme{Darmstadt}
%\usetheme{Dresden}
%\usetheme{Frankfurt}
%\usetheme{Goettingen}
%\usetheme{Hannover}
%\usetheme{Ilmenau}
%\usetheme{JuanLesPins}
%\usetheme{Luebeck}
\usetheme{Madrid}
%\usetheme{Malmoe}
%\usetheme{Marburg}
%\usetheme{Montpellier}
%\usetheme{PaloAlto}
%\usetheme{Pittsburgh}
%\usetheme{Rochester}
%\usetheme{Singapore}
%\usetheme{Szeged}
%\usetheme{Warsaw}

% As well as themes, the Beamer class has a number of color themes
% for any slide theme. Uncomment each of these in turn to see how it
% changes the colors of your current slide theme.

%\usecolortheme{albatross}
%\usecolortheme{beaver}
%\usecolortheme{beetle}
%\usecolortheme{crane}
%\usecolortheme{dolphin}
%\usecolortheme{dove}
%\usecolortheme{fly}
%\usecolortheme{lily}
%\usecolortheme{orchid}
%\usecolortheme{rose}
%\usecolortheme{seagull}
%\usecolortheme{seahorse}
%\usecolortheme{whale}
%\usecolortheme{wolverine}

%\setbeamertemplate{footline} % To remove the footer line in all slides uncomment this line
%\setbeamertemplate{footline}[page number] % To replace the footer line in all slides with a simple slide count uncomment this line

%\setbeamertemplate{navigation symbols}{} % To remove the navigation symbols from the bottom of all slides uncomment this line
}

% remove title and author from left panel
%  \makeatletter
%   \setbeamertemplate{sidebar \beamer@sidebarside}%{sidebar theme}
%   {
%     \beamer@tempdim=\beamer@sidebarwidth%
%     \advance\beamer@tempdim by -6pt%
%     \insertverticalnavigation{\beamer@sidebarwidth}%
%     \vfill
%     \ifx\beamer@sidebarside\beamer@lefttext%
%     \else%
%       \usebeamercolor{normal text}%
%       \llap{\usebeamertemplate***{navigation symbols}\hskip0.1cm}%
%       \vskip2pt%
%     \fi%
%   }%
% \makeatother
% done remove title and author from left panel 

\hypersetup{colorlinks,citecolor=red}
\usepackage{graphicx} % Allows including images
\usepackage{booktabs} % Allows the use of \toprule, \midrule and \bottomrule in tables
\usepackage{natbib}
\usepackage{apalike}
\usepackage{comment}
% \usepackage{enumitem}
% \setlist[itemize]{topsep=0pt,before=\leavevmode\vspace{-1.5em}}
% \setlist[description]{style=nextline}
\usepackage{amsthm}
\usepackage{media9}
% \usepackage{multimedia}
\usepackage{caption}
\usepackage{subcaption}
\usepackage{hyperref}
\usepackage{tikz}
\tikzset{
     arrow/.style={-{Stealth[]}}
     }
\usetikzlibrary{positioning,arrows.meta}
\usetikzlibrary{shapes.geometric}

\setbeamertemplate{navigation symbols}{}%remove navigation symbols
\setbeamertemplate{caption}[numbered]%allow figure numbers

\usepackage{setspace}

% \newtheorem{example}{Example}
% \setbeamertemplate{theorems}[numbered]

\newenvironment<>{example1}[1][Example 1]{%
  \setbeamercolor{block title}{fg=white,bg=cyan!75!black}%
  \begin{block}{#1}}{\end{block}}
\newenvironment<>{example2}[1][Example 2]{%
  \setbeamercolor{block title}{fg=white,bg=magenta!75!black}%
  \begin{block}{#1}}{\end{block}}

\newcounter{saveenumi}
\newcommand{\seti}{\setcounter{saveenumi}{\value{enumi}}}
\newcommand{\conti}{\setcounter{enumi}{\value{saveenumi}}}
\newcommand{\keepi}{\addtocounter{saveenumi}{-1}\setcounter{enumi}{\value{saveenumi}}}

%----------------------------------------------------------------------------------------
%	TITLE PAGE
%----------------------------------------------------------------------------------------

\title{Spectral analysis}

\author{Joaqu\'{i}n Rapela} % Your name
\institute[Gatsby Unit, UCL] % Your institution as it will appear on the bottom of every slide, may be shorthand to save space
{
Gatsby Computational Neuroscience Unit\\University College London % Your institution for the title page
}
\date{\today} % Date, can be changed to a custom date

\AtBeginSection[]
  {
     \begin{frame}<beamer>
     \frametitle{Contents}
         \tableofcontents[currentsection,hideallsubsections]
     \end{frame}
  }

\begin{document}

\begin{frame}
\titlepage % Print the title page as the first slide
\end{frame}

% \begin{frame}
% \frametitle{Contents} % Table of contents slide, comment this block out to remove it
% \tableofcontents % Throughout your presentation, if you choose to use \section{} and \subsection{} commands, these will automatically be printed on this slide as an overview of your presentation
% \end{frame}

\begin{frame}
    \frametitle{Motivation}

    \begin{itemize}

        \item We are monitoring \textbf{continuous} signal.

        \item We are saving \textbf{discrete} samples of this signal in our
            computer.

        \item \textcolor{red}{Under what conditions, and how, can we recover the continuous
            signal $x(t)$ from the saved samples?}

    \end{itemize}

    \begin{center}
        \includegraphics[width=3in]{figures/sampling.png}
    \end{center}

\end{frame}

\begin{frame}
    \frametitle{The Fourier Transform}

    The Fourier transform allows to represent a continuous signal as a linear
    combination of sinusoids.

    \begin{center}
        \includegraphics[height=1in]{figures/squareWave.png}\linebreak
        \includegraphics[height=1.5in]{figures/squareWaveAsSumOfSinusoids.png}
    \end{center}

\end{frame}

\begin{frame}
    \frametitle{The Fourier Transform}

    \begin{align*}
        x(t)&=\frac{1}{2\pi}\int_{-\infty}^\infty X(\omega)e^{\omega t}\
        d\omega=\frac{1}{2\pi}\int_{-\infty}^\infty X(\omega)(\cos(\omega
        t)+j\sin(\omega t))\ d\omega\\
            &=\frac{1}{2\pi}\int_{-\infty}^\infty X(\omega)\cos(\omega t)
            d\omega +
            j\frac{1}{2\pi}\int_{-\infty}^\infty X(\omega)\sin(\omega t)\
            d\omega\\
            \text{where}&\\
        x(\omega)&=\int_{-\infty}^\infty x(t)e^{-j\omega t}\ dt
    \end{align*}

\end{frame}

\begin{frame}
    \frametitle{Sampled signal}

    We sample values of a continuous function $x(t)$ at regular times $nT$,
    where $T$ is called the \textbf{sampling interval} and its inverse $1/T$ is
    called the \textbf{sampling frequency}.

    \begin{center}
        \includegraphics[height=2.0in]{figures/sampledSignal.png}
    \end{center}
    \hfill\small\citet{porat97}

\end{frame}

\begin{frame}
    \frametitle{Continuous representation of a sampled signal}

    \begin{itemize}

        \item Dirac delta function ($\delta(t))$:

            \begin{align*}
                \delta(t-t_0)&=0\quad t\neq t_0\\
                \int_{t_1}^{t_2}f(t)\delta(t-t_0)\,dt&=f(t_0)\quad t_1<t_0<t_2
            \end{align*}

        \item Thus, a sampled signal (with sample interval $T$) can be represented with
            the function:

            \begin{align*}
                x_s(t)=\sum_{n=-\infty}^\infty x(t)\,\delta(t-nT)
            \end{align*}

            \begin{itemize}

                \item $x_s(t)=0\quad t\neq nT$

                \item $\int_{nT-\Delta}^{nT+\Delta}x_s(t)\,dt=x(nT)$

            \end{itemize}

    \end{itemize}
\end{frame}

\begin{frame}
    \frametitle{Sampling theorem}

    \begin{theorem}[Sampling theorem \citep{shannon48}]

        \small

        Let $x_s(t)$ be a sampled signal, with sampling period $T$,
        of a continuous signal $x(t)$.

        Let $X(\omega)$ be the Fourier transform of $x(t)$.

        Then the Fourier
        transform of $x_s(t)$ is
        $X_s(\omega)=\frac{1}{T}\sum_{k=-\infty}^\infty
        X\left(\omega-\frac{2\pi k}{T}\right)$.

        \normalsize
    \end{theorem}

\end{frame}

\begin{frame}
    \frametitle{Sampling theorem}

    \begin{center}
        \includegraphics[height=2.75in]{figures/samplingTheorem.pdf}
        $\omega_s=\frac{2\pi}{T_s}$
    \end{center}

    \hfill\small\citet{poularikasAndSeely94}

\end{frame}

\begin{frame}
    \frametitle{Reconstruction procedure}

    \begin{enumerate}

        \item Fourier transform the sampled signal $x_s(t)$, yielding
            $X_s(\omega)$.

            \begin{center}
                \includegraphics[height=.5in]{figures/originalAndFTsampled.pdf}
            \end{center}

        \item low-pass filter $X_s(\omega)$ between frequencies
            $\omega=-\frac{\pi}{T}$ and $\omega=\frac{\pi}{T}$, yielding
            $X_s^{LP}(\omega)$.

            \begin{center}
                \includegraphics[height=.5in]{figures/lpFTSampled.pdf}
            \end{center}

        \item the desired continuous signal is the inverse Fourier transform of
            $X_s^{LP}(\omega)$ (i.e., $x(t)=\mathcal{IFT}\{X_s^{LP}(\omega)\}$).

            \begin{center}
                \includegraphics[height=.5in]{figures/iftLPFTSampled2.pdf}
            \end{center}

    \end{enumerate}

\end{frame}

\begin{frame}
    \frametitle{Reconstruction requirements}

    \begin{itemize}

        \item we want

            \begin{center}
                \includegraphics[height=0.75in]{figures/ftSampled.pdf}
            \end{center}

        \item we do NOT want 

            \begin{center}
                \includegraphics[height=0.75in]{figures/aliasing.pdf}
            \end{center}

        \item we want $\frac{\omega_s}{2}>\omega_N$, or
            $\frac{2\pi}{T}\frac{1}{2}=\frac{\omega_s}{2}>\omega_N=2\pi f_N$,
            or $\frac{1}{T}\frac{1}{2}>f_N$, or
            $f_s=\frac{1}{T}>2f_N$, or
            $\textcolor{red}{f_s>2f_N}$.

            \onslide<2->{\textcolor{red}{To reconstruct without error a
            continuous signal from its samples we need to sample it at a
            frequency larger than twice its maximal frequency.}}

    \end{itemize}

\end{frame}

\begin{frame}
    \frametitle{Reconstruction example}

    \begin{itemize}

        \item continuous signal in time domain
            \begin{align*}
                x(t)&=\left(sinc(f_0 t)\right)^2\\
                sinc(t)&=\left\{\begin{array}{l l}
                                    \frac{sin(\pi t)}{\pi t} & t\neq 0\\
                                    1 & t=0
                                \end{array}\right .
            \end{align*}

        \item continuous signal in frequency domain
            \begin{align*}
                x(\omega)&=\left\{\begin{array}{l l}
                    \frac{1}{f_0}\left(1-\frac{|\omega|}{2\pi f_0}\right) & \omega\leq 2\pi f_0\\
                    0 & \omega>\pi f_0\\
                                   \end{array}\right .
            \end{align*}
            $\omega_N=2\pi f_0$, $f_N=f_0$ and by the sampling  theorem $f_s>2f_0$.
    \end{itemize}

\end{frame}

\begin{frame}
    \frametitle{Reconstruction example: sampling below the Nyquist rate}

    \begin{center}
        \includegraphics[height=2.5in]{figures/exampleSampledBelowNyquistRate.pdf}
    \end{center}
    \hfill\small\citet{porat97}
\end{frame}

\begin{frame}
    \frametitle{Reconstruction example: sampling below the Nyquist rate}

    \begin{center}
        \includegraphics[height=3in]{figures/example_NiquistFactor0.80.png}
    \end{center}

\end{frame}

\begin{frame}
    \frametitle{Reconstruction example: sampling at the Nyquist rate}

    \begin{center}
        \includegraphics[height=3in]{figures/example_NiquistFactor1.00.png}
    \end{center}

\end{frame}

\begin{frame}
    \frametitle{Summary}

\end{frame}

\begin{frame}
    \frametitle{Bibliography}

    \bibliographystyle{apalike}
    \bibliography{communications,signalProcessing}

\end{frame}

\end{document}
