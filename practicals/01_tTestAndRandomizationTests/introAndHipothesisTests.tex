
\documentclass{beamer}

\mode<presentation> {

% The Beamer class comes with a number of default slide themes
% which change the colors and layouts of slides. Below this is a list
% of all the themes, uncomment each in turn to see what they look like.

%\usetheme{default}
%\usetheme{AnnArbor}
%\usetheme{Antibes}
%\usetheme{Bergen}
%\usetheme{Berkeley}
%\usetheme{Berlin}
%\usetheme{Boadilla}
%\usetheme{CambridgeUS}
%\usetheme{Copenhagen}
%\usetheme{Darmstadt}
%\usetheme{Dresden}
%\usetheme{Frankfurt}
%\usetheme{Goettingen}
%\usetheme{Hannover}
%\usetheme{Ilmenau}
%\usetheme{JuanLesPins}
%\usetheme{Luebeck}
\usetheme{Madrid}
%\usetheme{Malmoe}
%\usetheme{Marburg}
%\usetheme{Montpellier}
%\usetheme{PaloAlto}
%\usetheme{Pittsburgh}
%\usetheme{Rochester}
%\usetheme{Singapore}
%\usetheme{Szeged}
%\usetheme{Warsaw}

% As well as themes, the Beamer class has a number of color themes
% for any slide theme. Uncomment each of these in turn to see how it
% changes the colors of your current slide theme.

%\usecolortheme{albatross}
%\usecolortheme{beaver}
%\usecolortheme{beetle}
%\usecolortheme{crane}
%\usecolortheme{dolphin}
%\usecolortheme{dove}
%\usecolortheme{fly}
%\usecolortheme{lily}
%\usecolortheme{orchid}
%\usecolortheme{rose}
%\usecolortheme{seagull}
%\usecolortheme{seahorse}
%\usecolortheme{whale}
%\usecolortheme{wolverine}

%\setbeamertemplate{footline} % To remove the footer line in all slides uncomment this line
%\setbeamertemplate{footline}[page number] % To replace the footer line in all slides with a simple slide count uncomment this line

%\setbeamertemplate{navigation symbols}{} % To remove the navigation symbols from the bottom of all slides uncomment this line
}

% remove title and author from left panel
%  \makeatletter
%   \setbeamertemplate{sidebar \beamer@sidebarside}%{sidebar theme}
%   {
%     \beamer@tempdim=\beamer@sidebarwidth%
%     \advance\beamer@tempdim by -6pt%
%     \insertverticalnavigation{\beamer@sidebarwidth}%
%     \vfill
%     \ifx\beamer@sidebarside\beamer@lefttext%
%     \else%
%       \usebeamercolor{normal text}%
%       \llap{\usebeamertemplate***{navigation symbols}\hskip0.1cm}%
%       \vskip2pt%
%     \fi%
%   }%
% \makeatother
% done remove title and author from left panel 

\hypersetup{colorlinks,citecolor=}
\usepackage{graphicx} % Allows including images
\usepackage{booktabs} % Allows the use of \toprule, \midrule and \bottomrule in tables
\usepackage{natbib}
\usepackage{apalike}
\usepackage{comment}
% \usepackage{enumitem}
% \setlist[itemize]{topsep=0pt,before=\leavevmode\vspace{-1.5em}}
% \setlist[description]{style=nextline}
\usepackage{amsthm}
\usepackage{media9}
% \usepackage{multimedia}
\usepackage{caption}
\usepackage{subcaption}
\usepackage{hyperref}
\usepackage{tikz}
\tikzset{
     arrow/.style={-{Stealth[]}}
     }
\usetikzlibrary{positioning,arrows.meta}
\usetikzlibrary{shapes.geometric}

\setbeamertemplate{navigation symbols}{}%remove navigation symbols
\setbeamertemplate{caption}[numbered]%allow figure numbers

\usepackage{setspace}

% \newtheorem{example}{Example}
% \setbeamertemplate{theorems}[numbered]

\newenvironment<>{example1}[1][Example 1]{%
  \setbeamercolor{block title}{fg=white,bg=cyan!75!black}%
  \begin{block}{#1}}{\end{block}}
\newenvironment<>{example2}[1][Example 2]{%
  \setbeamercolor{block title}{fg=white,bg=magenta!75!black}%
  \begin{block}{#1}}{\end{block}}

\newcounter{saveenumi}
\newcommand{\seti}{\setcounter{saveenumi}{\value{enumi}}}
\newcommand{\conti}{\setcounter{enumi}{\value{saveenumi}}}
\newcommand{\keepi}{\addtocounter{saveenumi}{-1}\setcounter{enumi}{\value{saveenumi}}}

%----------------------------------------------------------------------------------------
%	TITLE PAGE
%----------------------------------------------------------------------------------------

\title{Introduction to hypothesis testing}

\author{Joaqu\'{i}n Rapela} % Your name
\institute[Gatsby Unit, UCL] % Your institution as it will appear on the bottom of every slide, may be shorthand to save space
{
Gatsby Computational Neuroscience Unit\\University College London % Your institution for the title page
}
\date{\today} % Date, can be changed to a custom date

\AtBeginSection[]
  {
     \begin{frame}<beamer>
     \frametitle{Contents}
         \tableofcontents[currentsection,hideallsubsections]
     \end{frame}
  }

\begin{document}

\begin{frame}
\titlepage % Print the title page as the first slide
\end{frame}

\begin{frame}
\frametitle{Contents} % Table of contents slide, comment this block out to remove it
\tableofcontents % Throughout your presentation, if you choose to use \section{} and \subsection{} commands, these will automatically be printed on this slide as an overview of your presentation
\end{frame}

\section{Course notes}

\begin{frame}
    \frametitle{Background}

    \begin{itemize}
        \item Last Spring 2023 I helped in the discussion sessions of this
            course.

        \item Suggested to Klara Olofsdotter (SWC PhD program coordinator) and Sonja Hofer (SWC PhD program faculty coordinator) to ask SWC PhD students to take this course. They liked the idea.

        \item I volunteered to lead discussions and do grading with Gatsby Unit
            PhD students and postdoctoral scholars.

    \end{itemize}

\end{frame}

\begin{frame}
\frametitle{A few motivations to run this course}

    \begin{enumerate}
        \item Gain more teaching experience.
        \item Provide SWC PhD students with essential neural data-analysis tools.
        \item Contribute to better interactions between the SWC and the Gatsby Unit.
    \end{enumerate}

\end{frame}

\begin{frame}
\frametitle{Course structure}

    \tiny
    \begin{center}
        \begin{tabular} { | l | l | l | l | l | }\hline
            Week 01 & Jan 11 & The t-test and randomisation tests & Joaquin Rapela &  tutorial \\\hline
            Week 02 & Jan 18 & Power spectra & Joaquin Rapela & tutorial \\
                    &        &               & Yousef Mohammadi & \\
                    &        &               & Joe Zimminski & \\\hline
            Week 03 & Jan 25 & Spectrograms  & Joaquin Rapela & tutorial \\
                    &        & and coherence & Yousef Mohammadi & \\
                    &        &               & Joe Zimminski & \\\hline
            Week 04 & Feb 01 & Circular statistics & Joaquin Rapela & tutorial \\\hline
            Week 05 & Feb 08 & Singular value decomposition & Will Dorrell & tutorial \\\hline\hline
            Week 06 & Feb 15 & Linear regression & Lior Fox & lecture\\
                    & Feb 16 &                   &          & tutorial\\\hline
            Week 07 & Feb 22 & Linear dynamical systems & Aniruddh Galgali & lecture \\
                    &        &                          & Joaquin Rapela & \\
                    & Feb 23 &                          &                  & tutorial\\\hline
            Week 08 & Feb 29 & no class (CoSyNe)        &                  & \\\hline
            Week 09 & Mar 07 & Artificial neural networks & Erin Grant & lecture\\
                    & Mar 08 &                            &            & tutorial\\\hline
            Week 10 & Mar 14 & Experimental control with Bonsai & Goncalo Lopes & lecture\\
                    &        &                                  & Joaquin Rapela & \\
                    & Mar 15 &                                  & & tutorial\\\hline
            Week 11 & Mar 21 & Reinforcement learning           & & lecture\\
                    & Mar 22 &                                  & & tutorial\\\hline
            Week 12 & Mar 28 & Project development & & \\
            Week 15 & Apr 25 &                     & & \\\hline 
            Week 16 & May 02 & Project presentations & & \\\hline
        \end{tabular}
    \end {center}
    Teaching assistants: Kira Dusterwald, Sihao (Daniel) Liu
    \normalsize





\end{frame}

\begin{frame}
    \frametitle{Graded worksheets}

    Every Thursday we will assign you a worksheet that is due on the second
    Monday after the assignment.

\end{frame}

\section{Statistical remarks}

\begin{frame}
    \frametitle{Working examples}

    \begin{example1}

        We know that the average running speed of control mice is 1~cm/sec. The
        sample average running speed of a cohort (n=100) of transgenic mice is
        $\bar{x}=2.7\,\text{cm/sec}$ and the sample standard deviation is
        $s=10~\text{cm/sec}$. Is the average running speed of mice in the
        transgenic cohort larger than that of control mice? Test with a
        confidence level $\alpha=0.01$.

    \end{example1}

\end{frame}

\begin{frame}
    \frametitle{Working examples}

    \begin{example2}

        We want to study the effect of a new drug on visual electrophysiology
        in humans. We know that the mean peak evoked response potential (ERP)
        over V1 during the first 200~ms after stimuli presentation is 2~mV. The
        sample mean peak ERP for a group of 50 medicated subjects is
        $\bar{x}=1.3\,\text{mV}$ and the sample standard deviation is
        $s=2.6~\text{mV}$. Does taking the new drug change the mean evoked ERP
        over V1? Provide your test p-value.

    \end{example2}

\end{frame}

\begin{frame}
\frametitle{Statistical remarks}

    \begin{enumerate}

        \item Random data (e.g., observed data) is characterised using
            probability distributions. For example a:

            \begin{itemize}

                \item Normal distribution with parameters mean $\mu$ and variance
                    $\sigma^2$, $\mathcal{N}(\mu,\sigma^2)$,

                    \begin{center}
                        \href{http://www.gatsby.ucl.ac.uk/~rapela/neuroinformatics/2024/htTutorial/figures/gaussians.html}{\includegraphics[width=2in]{figures/gaussians.png}}
                    \end{center}

                \item Exponential distribution with rate parameter $\lambda$,
                    $\mathcal{E}(\lambda)$,

                \item Poisson distribution with expected rate parameter $\lambda$,
                    $\mathcal{P}(\lambda)$,

                \item Binomial distribution with number of observation parameter $n$
                    and with a success probability parameter $p$,
                    $\mathcal{B}(n,p)$.

            \end{itemize}
            \seti
    \end{enumerate}


\end{frame}

\begin{frame}
\frametitle{Statistical remarks}

    \begin{enumerate}

        \conti

        \item One branch of statistics, \textbf{estimation theory}, provides
            tools to estimate parameters of distributions from observations.

        \item Another branch of statistics, \textbf{hypothesis testing},
            provides tools to make statistically-informed decisions about
            values of parameters of distributions.

            \seti

    \end{enumerate}

\end{frame}

\begin{frame}
\frametitle{Statistical remarks}

    \begin{enumerate}[<+->]

        \conti

		\item To estimate parameters, or to make decisions about them, we use
observations, $x_1,\ldots,x_N$, that are \textbf{independent and identically
distributed}.

            \begin{example1}

				An observation is the average speed of a transgenic mouse
during an experimental session. We assume that the average speeds of all
mice are samples from a common probability density function (identically
distributed) and that average speeds are independent across mice (independent).

            \end{example1}

            \begin{example2}

				An observation is the peak ERP of a medicated cohort
subject. We assume that these ERPs are samples from the same probability density
function (identically distributed) and that these ERPs are independent across
subjects (independent).

            \end{example2}

            \seti

    \end{enumerate}

\end{frame}

\begin{frame}
\frametitle{Statistical remarks}

    \begin{enumerate}

        \conti

		\item A goal of statistics is to \textbf{infer properties of the
population} (e.g., the effect of the genetic manipulation on the running speed
of mice) from \textbf{properties of the sample} (e.g., the effect of the
manipulation on the running speed of the 100 sampled mice).

		\seti

    \end{enumerate}

\end{frame}

\begin{frame}
\frametitle{Statistical remarks}

	\begin{theorem}[Central Limit Theorem]
        Let $X_1,\ldots,X_n$ be independent and identically distributed random
        variables with mean $\mu$ and finite variance $\sigma^2$. Let $n$ be
        large. Then the sample mean
        \begin{align}
            \bar{X}=\frac{1}{n}\sum_{i=1}^NX_i\label{eq:sampleMean}
        \end{align}
        is distributed as $\bar{X}\sim\mathcal{N}(\mu,\frac{\sigma^2}{n})$.
	\end{theorem}

    Note: if $\sigma^2$ is unknown, we can estimate $\sigma^2$ with the sample
    variance $s^2=\frac{1}{n-1}\sum_{i=1}^n(x_i-\bar{x})^2$.  Then, for large
    $n$, $\bar{X}$ is approximately distributed as
    $\mathcal{N}(\mu,\frac{s^2}{n})$.

\end{frame}

\section{Hypothesis testing}

\begin{frame}
\frametitle{Main source}

    Chapter 9 ``Large-sample test of hypothesis'' and chapter 10 ``Inference from
    small samples'' from 

    \begin{center}
        \includegraphics[height=2.5in]{figures/mendenhall09-introductionToProbabilityAndStatistics-13edition_cover.pdf}
    \end{center}

\end{frame}

\begin{frame}
\frametitle{Null and alternative hypothesis}

    \begin{itemize}

        \item In hypothesis testing we work with a \textbf{null hypothesis},
            $\mathcal{H}_0$, and an \textbf{alternative hypothesis},
            $\mathcal{H}_a$, collect a sample of data $x_1,\ldots,x_N$, and
            test if this data provides sufficient statistical evidence in favour
            of the alternative hypothesis. If this happens we reject the null
            hypothesis.

        \item However, if the collected data does not provide sufficient
            statistical evidence in favour of the alternative hypothesis, we do
            not accept the null hypothesis, but we say that we failed to reject
            it.  \textbf{Hypothesis tests do not prove null hypothesis, they
            only provide statistical evidence to reject it, or fail to reject it.}

    \end{itemize}

	\onslide<2->{
	\begin{alertblock}

		The hypothesis that we aim to prove should be the alternative one.

	\end{alertblock}
	}

\end{frame}

\begin{frame}
\frametitle{Null and alternative hypothesis}

    \begin{example1}

        Is the average running speed of the transgenic cohort larger than that of the control cohort (i.e., 2~cm/sec)?

        \begin{description}

            \item[$\mathcal{H}_0$]: the average running speed of the transgenic cohort is 2~cm/sec.

            \item[$\mathcal{H}_a$]: the average running speed of the transgenic cohort is lager than 2~cm/sec.

        \end{description}

    \end{example1}

\end{frame}

\begin{frame}
\frametitle{Null and alternative hypothesis}

    \begin{example2}

		Is the mean peak visual ERP in the first 200~ms post stimuli different
in medicated than in control subjects (i.e., 2~mV)?

        \begin{description}

            \item[$\mathcal{H}_0$]: the mean peak visual ERP in medicated subjects is 2~mV.

            \item[$\mathcal{H}_a$]: the mean peak visual ERP in medicated subjects is different from 2~mV.

        \end{description}

    \end{example2}

\end{frame}

\begin{frame}
\frametitle{One- and two-tailed tests of hypothesis}

	\begin{description}

		\item[One-tailed test of hypothesis] directionality is suggested by the
alternative hypothesis. 

			\begin{example1}

				It is a one-tailed hypothesis test because the alternative
hypothesis requires that the mean speed of the transgenic mice be larger
(directionality) than that of the control mice.

			\end{example1}

		\onslide<2->{
		\item[Two-tailed test of hypothesis] directionality is not suggested by
the alternative hypothesis.

			\begin{example2}

				 It is a two-tailed hypothesis test because the alternative
hypothesis requires that the visual ERP of the medicated subjects be different
(no directionality) than that of the control subjects.

			\end{example2}
		}
	\end{description}

\end{frame}

\begin{frame}
\frametitle{Test statistic and its sampling distribution}

    \begin{itemize}

        \item To perform a hypothesis test we propose a \textbf{test statistic}, a
            function of the sample data, like the sample mean in
            Eq.~\ref{eq:sampleMean}.

        \item Because the sample data is random, the test statistic is also
            random. To perform hypothesis tests we need to know the
            distribution of the test statistic, which is called the
            \textbf{sampling distribution}.

    \end{itemize}

\end{frame}

\begin{frame}
\frametitle{Test statistic and its sampling distribution for the working
    examples}

	Because both examples are tests for the population mean, $\mu$, because the
sample mean, $\bar{x}$, is a good estimator of the population mean, and because
both examples use a large number of samples, we will use the standardised
sample mean as our test statistic:

	\begin{align*}
		Z = \frac{\bar{X}-\mu_0}{s/\sqrt{n}}
	\end{align*}

	From the central limit theorem, we know the sampling distribution of this
    test statistic under the null hypothesis:

	\begin{align*}
		 Z\sim\mathcal{N}(0, 1)
	\end{align*}

\end{frame}

\begin{frame}
\frametitle{Rejection region}

    The \textbf{reject region} is a region of low probability under the null
    hypothesis, which is consistent with alternative hypothesis. The
    \textbf{non-reject region}, is a region of large probability under the null
    hypothesis, that is inconsistent with the alternative hypothesis.

\end{frame}

\begin{frame}
\frametitle{Rejection region}

	\begin{example1}

        We want to reject the null hypothesis that the average running speed of
        the transgenic mice equals 2~cm/sec if their sample mean speed is much
        larger than 2~cm/sec.  That is, we want to reject the null hypothesis
        if $z$ is large and positive. How large is large?

        \vspace{0.1in}

        To answer this question we define the \textbf{Type I error} of a test,
        as the error of rejecting the null hypothesis when it is valid. We also
        define the \textbf{significance level of the hypothesis test},
        $\alpha$, as the probability of Type I error admitted by the test.

        \vspace{0.1in}

        When designing a test we first decide on its significance level
        $\alpha$.  We then reject the null hypothesis if $z$ is larger than the
        value $z_\alpha$ that leaves $\alpha$ probability to its right, We call
        this value the \textbf{critical value} of the test (see figure on next
        slide).

	\end{example1}

\end{frame}

\begin{frame}
\frametitle{Rejection region}

	\begin{example1}

    	\begin{center}
        	\includegraphics[width=4in]{figures/right_tailed_test.png}
    	\end{center}
        \hfill Mendenhall et al., 2009

	\end{example1}

\end{frame}

\begin{frame}
\frametitle{Rejection region}

	\begin{example2}

        We want to reject the null hypothesis that the drug has not effect on
        the average visually evoked ERP if the sample mean visually evoked ERP
        of medicated subjects is much smaller or much larger than the mean
        visually evoked ERP of control subjects (2~mV).
        How small is small and how large is large?

        \vspace{0.1in}

        We will reject the null hypothesis if the standardised mean, $z$, is
        larger than the value $z_{\alpha/2}$ that leaves $\alpha/2$ probability
        to its right or smaller than the value $-z_{\alpha/2}$ that leaves
        $\alpha/2$ probability to its left (see figure on next slide).

	\end{example2}

\end{frame}

\begin{frame}
\frametitle{Rejection region}

	\begin{example1}

    	\begin{center}
        	\includegraphics[width=4in]{figures/two_tailed_test.png}
    	\end{center}
        \hfill Mendenhall et al., 2009

	\end{example1}

\end{frame}

\begin{frame}
    \frametitle{Large-sample hypothesis test for the mean}

    \begin{center}
        \includegraphics[width=3.5in]{figures/large_sample_hypothesis_test.png}
    \end{center}
    \hfill \scriptsize{Mendenhall et al., 2009}

\end{frame}

\begin{frame}
\frametitle{Complete hypothesis test for example 1}

    \begin{example1}

        We know that the average running speed of control mice is 1~cm/sec. The
        sample average running speed of a cohort (n=100) of transgenic mice is
        $\bar{x}=2.7\,\text{cm/sec}$ and the sample standard deviation is
        $s=10~\text{cm/sec}$. Is the average running speed of mice in the
        transgenic cohort larger than that of control mice? Test with a
        confidence level $\alpha=0.01$.

        Relevant quantities: $\mu_0=1$, $n=100$, $\bar{x}=2.7$, $s=10$,
        $\alpha=0.01$.

        \begin{enumerate}

            \item identify the null hypothesis $H_0$: \textcolor{cyan}{the average running speed of the transgenic cohort is 2 cm/sec.}

            \item identify the alternative hypothesis $H_a$: \textcolor{cyan}{the average running speed of the transgenic cohort is larger than 2 cm/sec}.

            \item select a test statistic: \textcolor{cyan}{standardised sample mean $Z$}.

            \seti

        \end{enumerate}

	\end{example1}

\end{frame}

\begin{frame}
\frametitle{Complete hypothesis test for example 1}

    \begin{example1}

        \begin{enumerate}

            \conti

            \item set the rejection region: \textcolor{cyan}{right-tailed hypothesis test, with critical value $z_{0.01}=2.3263$}.

            \item calculate the observed value of the test statistic.

                \begin{align*}
                    \color{cyan}
                    z_\text{obs}=\frac{\bar{x}-\mu_0}{s/\sqrt{n}}=\frac{2.7-1}{10/\sqrt{100}}=1.7
                \end{align*}

            \item draw your conclusion: \textcolor{cyan}{$z_\text{obs}=1.7<2.3263=z_{0.01}$. Thus, there is not enough statistical evidence to reject the null hypothesis with a confidence level $\alpha=0.01$.}

        \end{enumerate}

        Would you reject the null hypothesis with  confidence level
        $\alpha=0.05$?

    \end{example1}

\end{frame}

\begin{frame}
\frametitle{p-value}

    \begin{definition}[p-value]

        The p-value (or observed significance level of a statistical test) is
        the probability that the test statistic is as extreme or more extreme
        than the observed test statistic value, when the null hypothesis is
        true.

    \end{definition}

    \begin{figure}
        \centering
        \begin{subfigure}[b]{0.45\textwidth}
            \centering
            \includegraphics[width=2in]{figures/right_sided_pvalue_colored.png}
            \caption{left-sided test}
        \end{subfigure}
        \hfill
        \begin{subfigure}[b]{0.45\textwidth}
            \centering
            \includegraphics[width=2in]{figures/two_sided_pvalue_colored.png}
            \caption{two-sided test}
        \end{subfigure}
        \caption{p-values are the area of the red-coloured regions}
    \end{figure}

\end{frame}

\begin{frame}
\frametitle{p-value}

    Notes:

    \begin{itemize}

        \item small p-values indicate that the probability of obtaining a test
            statistic as extreme or more extreme as the observed test statistic value
            is small, showing that our data supports the alternative hypothesis.

        \item large p-values show that our data does not support the
            alternative hypothesis.

        \item if the p-value is smaller than a pre-assigned confidence level
            $\alpha$, then the null hypothesis can be rejected and you can say
            that the result is statistically significant at confidence level
            $\alpha$.

    \end{itemize}

\end{frame}

\begin{frame}
\frametitle{Complete hypothesis test for example 2}

    \begin{example2}
        Worksheet problem 1.
    \end{example2}

\end{frame}

\begin{frame}
\frametitle{Taxonomy of hypothesis tests for the the mean}

    \begin{itemize}
        \item The previous tests worked well because the sample statistic for
            the sample mean was approximately normal. This happens when

            \begin{itemize}

                \item samples, of any size, are Gaussian, with known variance.

                \item the number of samples is sufficiently large (e.g.,
                    $n>30$) and samples are independent and identically
                    distributed from any distribution with mean $\mu$ and
                    finite variance $\sigma^2$, known or estimated by the
                    sample variance $s^2$ (central limit theorem).

            \end{itemize}

        \item What if samples are Gaussian, with unknown variance and we
            can only collect a small number $n$ of samples?

            Then the sample distribution of the mean follows a student-t
            distribution with n-1 degrees of freedom.

        \item What if samples are non-Gaussian and we can only collect a small
            number $n$ of samples?

            Use a resampling method (e.g., bootstrap of permutation test).
    \end{itemize}

\end{frame}

\begin{frame}
    \frametitle{Student-t distribution}

    The Student-t distribution:

    \begin{itemize}

        \item has a similar shape as the standard Normal distribution,

        \item has heavier tails than the standard Normal distribution,

        \item approaches the standard normal distribution as the number of
            degrees of freedom increase.
    \end{itemize}

    \begin{center}
        \href{https://www.gatsby.ucl.ac.uk/~rapela/neuroinformatics/2024/htTutorial/figures/studentTVsgaussians.html}{\includegraphics[width=2.5in]{figures/studentTVsgaussians.png}}
    \end{center}

\end{frame}

\begin{frame}
    \frametitle{Small-sample hypothesis test for the mean (t-test)}

    \begin{center}
        \includegraphics[width=2.5in]{figures/small_sample_hypothesis_test.png}
    \end{center}
    \hfill \scriptsize{Mendenhall et al., 2009}

\end{frame}

\begin{frame}
    \frametitle{Two types of errors}

    \begin{definition}[Type I error]

        The error of rejecting the null hypothesis when it is true is called
        type I error. The probability of type I error is denoted by the symbol
        $\alpha$ (Figure~\ref{fig:xBarUnderH0AndHa}).

    \end{definition}

    \begin{definition}[Type II error]

        The error of not rejecting the null hypothesis when the alternative one
        is true is called type II error. The probability of type II error is
        denoted by the symbol $\beta$ (Figure~\ref{fig:xBarUnderH0AndHa}).

    \end{definition}

    \begin{definition}[Power of a statistical test]

        The power of a statistical test is the probability of rejecting the
        null hypothesis when the alternative one is true and it is equal to
        $1-\beta$.

    \end{definition}

\end{frame}

\begin{frame}
    \frametitle{Calculation of $\beta$}

    \tiny
    \begin{example1}
        Calculate $\beta$ for the alternative hypothesis $\mathcal{H}_a:
        \mu_a=5.0$ (using, as before, $\mathcal{H}_0: \mu_0=1.0$, $\alpha=0.01$,
        $\bar{x}=2.7$, $s=10$, and $n=100$). See
        Figure~\ref{fig:xBarUnderH0AndHa}.

        We will first express the rejection region in terms of $\bar{X}$.
        Before we calculated the rejection region in terms of $Z$ as
        $Z>z_{\alpha}$ which is

        \begin{align*}
            z_\alpha<Z=\frac{\bar{X}-\mu_0}{s/\sqrt{n}}
        \end{align*}

        Then the rejection region in therms of $\bar{X}$ is

        \begin{align*}
            \bar{X}>\mu_0+z_\alpha\,s/\sqrt{n}
        \end{align*}

        Now

        \begin{align*}
            \beta&=P(\text{not reject}\ \mathcal{H}_0|\mathcal{H}_a\ \text{is
            true})=P(\bar{X}<\mu_0+z_\alpha\,s/\sqrt{n}|\mu=\mu_a)=P(\frac{\bar{X}-\mu_a}{s/\sqrt{n}}<\frac{\mu_0-\mu_a}{s/\sqrt{n}}+z_\alpha|\mu=\mu_a)\\
                 &=P(Z<\frac{\mu_0-\mu_a}{s/\sqrt{n}}+z_\alpha)=P(Z<\frac{1.0-2.0}{10/\sqrt{100}}+2.326)=P(Z<-1.6736)=0.047
        \end{align*}

    \end{example1}

    The worksheet contains an optional exercise illustrating how to find the
    minimum sample size $n$ to achieve a given probability of type I error
    $\alpha$ and probability of type II error $\beta$ for a given effect size.

    \normalsize
\end{frame}

\begin{frame}
    \frametitle{Calculation of $\beta$}
    \begin{center}
        \begin{figure}
            \includegraphics[width=3in]{figures/xBarUnderH0AndHa_colored.png}

             \caption{Probabilities of type I and type II errors ($\alpha$ and
             $\beta$, respectively) for the example in the previous page.}
             \label{fig:xBarUnderH0AndHa}

        \end{figure}
    \end{center}
\end{frame}

\begin{frame}
    \frametitle{Lecture notes and worksheet}

    They can both be found in the class repository at
    \href{https://github.com/joacorapela/neuroinformatics24}{https://github.com/joacorapela/neuroinformatics24}

    The lecture notes appear
    \href{https://github.com/joacorapela/neuroinformatics24/blob/master/practicals/01\_tTestAndRandomizationTests/introAndHipothesisTests.pdf}{here}
    and the worksheets appear
    \href{https://github.com/joacorapela/neuroinformatics24/blob/master/worsheets/01\_tTestAndRandomizationTests/worksheet\_tTestAndRandomizationTests.pdf}{here}.

\end{frame}

\begin{frame}
    \frametitle{Summary}

\end{frame}

\end{document}
