
\documentclass{beamer}

\mode<presentation> {

% The Beamer class comes with a number of default slide themes
% which change the colors and layouts of slides. Below this is a list
% of all the themes, uncomment each in turn to see what they look like.

%\usetheme{default}
%\usetheme{AnnArbor}
%\usetheme{Antibes}
%\usetheme{Bergen}
%\usetheme{Berkeley}
%\usetheme{Berlin}
%\usetheme{Boadilla}
%\usetheme{CambridgeUS}
%\usetheme{Copenhagen}
%\usetheme{Darmstadt}
%\usetheme{Dresden}
%\usetheme{Frankfurt}
%\usetheme{Goettingen}
%\usetheme{Hannover}
%\usetheme{Ilmenau}
%\usetheme{JuanLesPins}
%\usetheme{Luebeck}
\usetheme{Madrid}
%\usetheme{Malmoe}
%\usetheme{Marburg}
%\usetheme{Montpellier}
%\usetheme{PaloAlto}
%\usetheme{Pittsburgh}
%\usetheme{Rochester}
%\usetheme{Singapore}
%\usetheme{Szeged}
%\usetheme{Warsaw}

% As well as themes, the Beamer class has a number of color themes
% for any slide theme. Uncomment each of these in turn to see how it
% changes the colors of your current slide theme.

%\usecolortheme{albatross}
%\usecolortheme{beaver}
%\usecolortheme{beetle}
%\usecolortheme{crane}
%\usecolortheme{dolphin}
%\usecolortheme{dove}
%\usecolortheme{fly}
%\usecolortheme{lily}
%\usecolortheme{orchid}
%\usecolortheme{rose}
%\usecolortheme{seagull}
%\usecolortheme{seahorse}
%\usecolortheme{whale}
%\usecolortheme{wolverine}

%\setbeamertemplate{footline} % To remove the footer line in all slides uncomment this line
%\setbeamertemplate{footline}[page number] % To replace the footer line in all slides with a simple slide count uncomment this line

%\setbeamertemplate{navigation symbols}{} % To remove the navigation symbols from the bottom of all slides uncomment this line
}

% remove title and author from left panel
%  \makeatletter
%   \setbeamertemplate{sidebar \beamer@sidebarside}%{sidebar theme}
%   {
%     \beamer@tempdim=\beamer@sidebarwidth%
%     \advance\beamer@tempdim by -6pt%
%     \insertverticalnavigation{\beamer@sidebarwidth}%
%     \vfill
%     \ifx\beamer@sidebarside\beamer@lefttext%
%     \else%
%       \usebeamercolor{normal text}%
%       \llap{\usebeamertemplate***{navigation symbols}\hskip0.1cm}%
%       \vskip2pt%
%     \fi%
%   }%
% \makeatother
% done remove title and author from left panel 

\hypersetup{colorlinks,citecolor=red}
\usepackage{graphicx} % Allows including images
\usepackage{booktabs} % Allows the use of \toprule, \midrule and \bottomrule in tables
\usepackage{natbib}
\usepackage{apalike}
\usepackage{comment}
% \usepackage{enumitem}
% \setlist[itemize]{topsep=0pt,before=\leavevmode\vspace{-1.5em}}
% \setlist[description]{style=nextline}
\usepackage{amsthm}
\usepackage{media9}
% \usepackage{multimedia}
\usepackage{caption}
\usepackage{subcaption}
\usepackage{hyperref}
\usepackage{tikz}
\tikzset{
     arrow/.style={-{Stealth[]}}
     }
\usetikzlibrary{positioning,arrows.meta}
\usetikzlibrary{shapes.geometric}

\setbeamertemplate{navigation symbols}{}%remove navigation symbols
\setbeamertemplate{caption}[numbered]%allow figure numbers

\usepackage{setspace}

% \newtheorem{example}{Example}
% \setbeamertemplate{theorems}[numbered]

\newenvironment<>{example1}[1][Example 1]{%
  \setbeamercolor{block title}{fg=white,bg=cyan!75!black}%
  \begin{block}{#1}}{\end{block}}
\newenvironment<>{example2}[1][Example 2]{%
  \setbeamercolor{block title}{fg=white,bg=magenta!75!black}%
  \begin{block}{#1}}{\end{block}}

\newcounter{saveenumi}
\newcommand{\seti}{\setcounter{saveenumi}{\value{enumi}}}
\newcommand{\conti}{\setcounter{enumi}{\value{saveenumi}}}
\newcommand{\keepi}{\addtocounter{saveenumi}{-1}\setcounter{enumi}{\value{saveenumi}}}

%----------------------------------------------------------------------------------------
%	TITLE PAGE
%----------------------------------------------------------------------------------------

\title{Phase analysis for the characeterization of ECoG recordings in human speech production}

\author{Joaqu\'{i}n Rapela} % Your name
\institute[Gatsby Unit, UCL] % Your institution as it will appear on the bottom of every slide, may be shorthand to save space
{
Gatsby Computational Neuroscience Unit\\University College London % Your institution for the title page
}
\date{\today} % Date, can be changed to a custom date

\AtBeginSection[]
  {
     \begin{frame}<beamer>
     \frametitle{Contents}
         \tableofcontents[currentsection,hideallsubsections]
     \end{frame}
  }

\begin{document}

\begin{frame}
\titlepage % Print the title page as the first slide
\end{frame}

% \begin{frame}
% \frametitle{Contents} % Table of contents slide, comment this block out to remove it
% \tableofcontents % Throughout your presentation, if you choose to use \section{} and \subsection{} commands, these will automatically be printed on this slide as an overview of your presentation
% \end{frame}

\begin{frame}{Behaviour: production of consonant-vowel syllables}

    \begin{center}
        \href{sounds/analog1From391To451.wav}{
            \includegraphics[width=4in]{figures/cvsListPage1Clipped.pdf}
        }
    \end{center}

\end{frame}

\begin{frame}{Recordings: high-density and large coverage ECoG}

    \begin{center}
        \includegraphics[width=4.8in]{figures/grid_layout_clipped_withAreas.eps}
    \end{center}

\end{frame}

\begin{frame}{Phase relationships across trials: inter-trial coherence (ITC)}

    ITC plots: Figures~1 and~2 in \href{https://arxiv.org/abs/1606.02372}{Entrainment of travelling waves to rhythmic motor
    acts} (page 3).

    Steps to compute the Inter-Trial Coherence (ITC):

    \begin{enumerate}
        \item perform time-frequency decompositions of several trials,
        \item for each time and frequency bin, build a circular histogram of
            phases
        \item plot the mean resultant length from the previous histogram.
    \end{enumerate}

    Circular statistics methods: Section A.1.1 Circular statistics concepts in
    \href{https://arxiv.org/abs/1606.02372}{Entrainment of travelling waves to
    rhythmic motor acts}.

\end{frame}

\begin{frame}{Phase relationships across electrodes: travelling waves (TWs)}

    TWs in time: Figure~6 in
    \href{https://arxiv.org/abs/1606.02372}{Entrainment of travelling waves to
    rhythmic motor acts} (page 9).

    Steps to finds travelling waves (TWs):

    \begin{enumerate}
        \item narrow bandpass LFPs around a frequency of interest (e.g.,
            0.62~Hz in the speech production examples).
        \item extract phases from the Hilbert transform the filtered LFPs as
            all electrodes.
        \item check if there exist a linear relationship between phases and
            electrode locations (see next point).
    \end{enumerate}

    Detection of travelling waves events: Figure~5 in
    \href{https://arxiv.org/pdf/1806.09559}{Travelling waves appear and
    disappear in unison with produced speech}.

\end{frame}

\begin{frame}{Phase relationships across electrodes: travelling waves (TWs)}

    Phase relations across all grid electrodes: Figure~16 in
    \href{https://arxiv.org/pdf/1705.01615}{Rhythmic production of
    consonant-vowel syllables synchronises travelling waves in speech-processing
    brain regions} (page 27, note caption).

    Phase alignment with the initiation of the production of a consonant-vowel
    syllable: Figure~10 in \href{https://arxiv.org/pdf/1705.01615}{Rhythmic
    production of consonant-vowel syllables synchronises travelling waves in
    speech-processing brain regions} (page 21, note caption).

\end{frame}

\begin{frame}
    \frametitle{Recommended readings}

    \begin{itemize}

        \item \citet{buzsaki06}

        \item \citet{winfree80}

    \end{itemize}
\end{frame}

\begin{frame}
    \frametitle{Summary}

\end{frame}

\begin{frame}
    \frametitle{Bibliography}

    \bibliographystyle{apalike}
    \bibliography{rhythms}

\end{frame}

\end{document}
