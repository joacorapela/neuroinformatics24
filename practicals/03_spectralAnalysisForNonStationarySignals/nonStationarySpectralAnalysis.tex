
\documentclass{beamer}

\mode<presentation> {

% The Beamer class comes with a number of default slide themes
% which change the colors and layouts of slides. Below this is a list
% of all the themes, uncomment each in turn to see what they look like.

%\usetheme{default}
%\usetheme{AnnArbor}
%\usetheme{Antibes}
%\usetheme{Bergen}
%\usetheme{Berkeley}
%\usetheme{Berlin}
%\usetheme{Boadilla}
%\usetheme{CambridgeUS}
%\usetheme{Copenhagen}
%\usetheme{Darmstadt}
%\usetheme{Dresden}
%\usetheme{Frankfurt}
%\usetheme{Goettingen}
%\usetheme{Hannover}
%\usetheme{Ilmenau}
%\usetheme{JuanLesPins}
%\usetheme{Luebeck}
\usetheme{Madrid}
%\usetheme{Malmoe}
%\usetheme{Marburg}
%\usetheme{Montpellier}
%\usetheme{PaloAlto}
%\usetheme{Pittsburgh}
%\usetheme{Rochester}
%\usetheme{Singapore}
%\usetheme{Szeged}
%\usetheme{Warsaw}

% As well as themes, the Beamer class has a number of color themes
% for any slide theme. Uncomment each of these in turn to see how it
% changes the colors of your current slide theme.

%\usecolortheme{albatross}
%\usecolortheme{beaver}
%\usecolortheme{beetle}
%\usecolortheme{crane}
%\usecolortheme{dolphin}
%\usecolortheme{dove}
%\usecolortheme{fly}
%\usecolortheme{lily}
%\usecolortheme{orchid}
%\usecolortheme{rose}
%\usecolortheme{seagull}
%\usecolortheme{seahorse}
%\usecolortheme{whale}
%\usecolortheme{wolverine}

%\setbeamertemplate{footline} % To remove the footer line in all slides uncomment this line
%\setbeamertemplate{footline}[page number] % To replace the footer line in all slides with a simple slide count uncomment this line

%\setbeamertemplate{navigation symbols}{} % To remove the navigation symbols from the bottom of all slides uncomment this line
}

% remove title and author from left panel
%  \makeatletter
%   \setbeamertemplate{sidebar \beamer@sidebarside}%{sidebar theme}
%   {
%     \beamer@tempdim=\beamer@sidebarwidth%
%     \advance\beamer@tempdim by -6pt%
%     \insertverticalnavigation{\beamer@sidebarwidth}%
%     \vfill
%     \ifx\beamer@sidebarside\beamer@lefttext%
%     \else%
%       \usebeamercolor{normal text}%
%       \llap{\usebeamertemplate***{navigation symbols}\hskip0.1cm}%
%       \vskip2pt%
%     \fi%
%   }%
% \makeatother
% done remove title and author from left panel 

\hypersetup{colorlinks,citecolor=red}
\usepackage{graphicx} % Allows including images
\usepackage{booktabs} % Allows the use of \toprule, \midrule and \bottomrule in tables
\usepackage{natbib}
\usepackage{apalike}
\usepackage{comment}
% \usepackage{enumitem}
% \setlist[itemize]{topsep=0pt,before=\leavevmode\vspace{-1.5em}}
% \setlist[description]{style=nextline}
\usepackage{amsthm}
\usepackage{media9}
% \usepackage{multimedia}
\usepackage{caption}
\usepackage{subcaption}
\usepackage{hyperref}
\usepackage{tikz}
\tikzset{
     arrow/.style={-{Stealth[]}}
     }
\usetikzlibrary{positioning,arrows.meta}
\usetikzlibrary{shapes.geometric}

\setbeamertemplate{navigation symbols}{}%remove navigation symbols
\setbeamertemplate{caption}[numbered]%allow figure numbers

\usepackage{setspace}

% \newtheorem{example}{Example}
% \setbeamertemplate{theorems}[numbered]

\newenvironment<>{example1}[1][Example 1]{%
  \setbeamercolor{block title}{fg=white,bg=cyan!75!black}%
  \begin{block}{#1}}{\end{block}}
\newenvironment<>{example2}[1][Example 2]{%
  \setbeamercolor{block title}{fg=white,bg=magenta!75!black}%
  \begin{block}{#1}}{\end{block}}

\newcounter{saveenumi}
\newcommand{\seti}{\setcounter{saveenumi}{\value{enumi}}}
\newcommand{\conti}{\setcounter{enumi}{\value{saveenumi}}}
\newcommand{\keepi}{\addtocounter{saveenumi}{-1}\setcounter{enumi}{\value{saveenumi}}}

%----------------------------------------------------------------------------------------
%	TITLE PAGE
%----------------------------------------------------------------------------------------

\title{Non-stationary spectral analysis}

\author{Joaqu\'{i}n Rapela} % Your name
\institute[Gatsby Unit, UCL] % Your institution as it will appear on the bottom of every slide, may be shorthand to save space
{
Gatsby Computational Neuroscience Unit\\University College London % Your institution for the title page
}
\date{\today} % Date, can be changed to a custom date

\AtBeginSection[]
  {
     \begin{frame}<beamer>
     \frametitle{Contents}
         \tableofcontents[currentsection,hideallsubsections]
     \end{frame}
  }

\begin{document}

\begin{frame}
\titlepage % Print the title page as the first slide
\end{frame}

% \begin{frame}
% \frametitle{Contents} % Table of contents slide, comment this block out to remove it
% \tableofcontents % Throughout your presentation, if you choose to use \section{} and \subsection{} commands, these will automatically be printed on this slide as an overview of your presentation
% \end{frame}

\section{Time-frequency uncertainty}

\begin{frame}
    \frametitle{Complex numbers}

    A complex number $a+ib$ is a vector in the complex plane.

    \begin{center}
        \includegraphics[width=3in]{figures/complexNumberGraphicalRepresentation.pdf}
    \end{center}

    \begin{itemize}

        \item $a$ and $b$ are the real and imaginary parts, respectively.

        \item $A$ is the magnitude.

        \item $\phi_X$ is the phase.

        \item using the \textbf{Euler's formula}
            $a+ib=A(\cos(\phi_X)+i\sin(\phi_X))=Ae^{i\phi_X}$.

    \end{itemize}

\end{frame}

\begin{frame}
    \frametitle{Four types of Fourier transforms}

    \tiny
    \begin{center}
        \begin{tabular}{|l || l | l || l | l|}
            \hline

            FT\footnote{Fourier transform} & continuous & $x(t)=\frac{1}{2\pi}\int x(j\Omega)\ e^{j\Omega t}\ d\Omega$ & continuous & $x(j\Omega)=\int_{-\infty}^\infty x(t)\ e^{-j\Omega t}\ dt$\\\hline

            FS\footnote{Fourier series} & periodic (T) &
            $x(t)=\sum_{k=-\infty}^\infty X[k]\ e^{j\frac{2\pi}{T}kt}$ &
            discrete (inf) & $X[k]=\int_{-T/2}^{T/2}x(t)\ e^{-j\frac{2\pi}{T}kt}\ dt$\\\hline

            DTFT\footnote{discrete time Fourier transform} & discrete (inf) &
            $x[n]=\frac{1}{2\pi}\int_{-\pi}^\pi X(\omega)\ e^{j\omega n}\
            d\omega$ & periodic ($2\pi$) & $X(\omega)=\sum_{n=-\infty}^\infty
            x[n]\ e^{-j\omega n}$\\\hline

            DFT\footnote{discrete Fourier transform} & discrete (finite) &
            $x[n]=\sum_{k=0}^{N-1}X[k]\ e^{j\frac{2\pi}{N}kn}$ &
            discrete (finite) & $X[k]=\sum_{n=0}^{N-1} x[n]\
            e^{-j\frac{2\pi}{N} nk}$\\\hline

        \end{tabular}
    \end{center}

    \footnotesize
    \begin{itemize}
        \item how does $\omega$ of the DTFT relates to $\Omega$ of the FT?
            \begin{align*}
                x[n]\sim x_s(t)&=\sum_{n=-\infty}^\infty x[n]\delta(t-nT)\\
                X_s(j\Omega)&=\int_{-\infty}^\infty x_s(t)e^{j\Omega
                t}dt=\int_{-\infty}^\infty\left(\sum_{n=-\infty}^\infty
                x[n]\delta(t-nT)\right)e^{j\Omega t}dt\\
                               &=\sum_{n=-\infty}^\infty
                               x[n]\left(\int_{-\infty}^\infty\delta(t-nT)e^{j\Omega
                               t}dt\right)=\sum_{n=-\infty}^\infty
                               x[n]e^{j\Omega
                               nT}=\left.X_s(\omega)\right|_{\textcolor{red}{\omega=\Omega
                               T}}
            \end{align*}
    \end{itemize}
    \normalsize

\end{frame}

\begin{frame}
    \frametitle{Four types of Fourier transforms}

    \tiny
    \begin{center}
        \begin{tabular}{|l || l | l || l | l|}
            \hline

            FT\footnote{Fourier transform} & continuous & $x(t)=\frac{1}{2\pi}\int x(j\Omega)\ e^{j\Omega t}\ d\Omega$ & continuous & $x(j\Omega)=\int_{-\infty}^\infty x(t)\ e^{-j\Omega t}\ dt$\\\hline

            FS\footnote{Fourier series} & periodic (T) &
            $x(t)=\sum_{k=-\infty}^\infty X[k]\ e^{j\frac{2\pi}{T}kt}$ &
            discrete (inf) & $X[k]=\int_{-T/2}^{T/2}x(t)\ e^{-j\frac{2\pi}{T}kt}\ dt$\\\hline

            DTFT\footnote{discrete time Fourier transform} & discrete (inf) &
            $x[n]=\frac{1}{2\pi}\int_{-\pi}^\pi X(\omega)\ e^{j\omega n}\
            d\omega$ & periodic ($2\pi$) & $X(\omega)=\sum_{n=-\infty}^\infty
            x[n]\ e^{-j\omega n}$\\\hline

            DFT\footnote{discrete Fourier transform} & discrete (finite) &
            $x[n]=\sum_{k=0}^{N-1}X[k]\ e^{j\frac{2\pi}{N}kn}$ &
            discrete (finite) & $X[k]=\sum_{n=0}^{N-1} x[n]\
            e^{-j\frac{2\pi}{N} nk}$\\\hline

        \end{tabular}
    \end{center}

    \footnotesize
    \begin{itemize}

        \item how to find the frequency in Hz corresponding to a DFT coefficient k?
            \begin{align*}
                \omega&=\frac{2\pi}{N}k\\
                \Omega=\frac{\omega}{T}&=2\pi\frac{k}{NT}\\
            \end{align*}
            Thus, the coefficient k corresponds to the frequency
            $\frac{k}{NT}$~Hz.

        \item what is the \textbf{frequency resolution} of a Fourier transform?

            It is the distance in Hz between two neighboring frequencies, i.e.,
            $\text{frequency resolution}=\frac{1}{NT}=\frac{1}{\text{signal duration}}$.
    \end{itemize}
    \normalsize

\end{frame}

\begin{frame}
    \frametitle{Time-frequency uncertainty}

    $\text{frequency resolution}=\frac{1}{NT}=\frac{1}{\text{signal duration}}$
    \footnotesize
    \begin{figure}
        \centering
        \begin{subfigure}[b]{0.35\textwidth}
            \centering
            \resizebox{2in}{!}{%
                \begin{tikzpicture}

  \draw[->] (-0.2,0) -- (3.5,0) node[right] {Time (sec)};
  \draw[->] (0,-0.2) -- (0,6.5) node[above] {Frequency (Hz)};

  \draw[-] (1.0,0.0) node[below]{1.0} -- (1.0,6.0);
  \draw[-] (2.0,0.0) node[below]{2.0} -- (2.0,6.0);
  \draw[-] (3.0,0.0) node[below]{3.0} -- (3.0,6.0);

  \draw[-] (0.0,2.0) node[left]{2.0} -- (3.0,2.0);
  \draw[-] (0.0,4.0) node[left]{4.0} -- (3.0,4.0);
  \draw[-] (0.0,6.0) node[left]{6.0} -- (3.0,6.0);

\end{tikzpicture}

            }%
            \caption{High temporal resolution}
        \end{subfigure}
        \hspace{.3in}
        \begin{subfigure}[b]{0.55\textwidth}
            \centering
            \resizebox{2.75in}{!}{%
                \begin{tikzpicture}

  \draw[->] (-0.2,0) -- (6.5,0) node[right] {Time (sec)};
  \draw[->] (0,-0.2) -- (0,3.5) node[above] {Frequency (Hz)};

  \draw[-] (2.0,0.0) node[below]{2.0} -- (2.0,3.0);
  \draw[-] (4.0,0.0) node[below]{4.0} -- (4.0,3.0);
  \draw[-] (6.0,0.0) node[below]{6.0} -- (6.0,3.0);

  \draw[-] (0.0,1.0) node[left]{1.0} -- (6.0,1.0);
  \draw[-] (0.0,2.0) node[left]{2.0} -- (6.0,2.0);
  \draw[-] (0.0,3.0) node[left]{3.0} -- (6.0,3.0);


\end{tikzpicture}

            }%
            \caption{High frequency resolution}
        \end{subfigure}
    \end{figure}

    \textcolor{red}{Tradeoff between frequency and time resolution}.
\end{frame}

\section{Spectral coherence interpretation}

\begin{frame}
    \frametitle{Spectral measures for multiple time series}

    \begin{itemize}

        \item The \textbf{cross-power} is

            \begin{align*}
                S_{XY}(f)&=\sum_{\tau=-\infty}^\infty R_{XY}(\tau)e^{-i2\pi f\tau}\\
                         &=X(f)Y^*(f)
            \end{align*}

        \item The \textbf{multi-trial spectral coherence} is

            \begin{align*}
                C_{XY}(f)&=\frac{\left|<S_{XY,k}(f)>\right|^2}{<S_{XX}(f)><S_{YY}(f)>}
            \end{align*}

        \item The \textbf{spectral coherence} is

            \begin{align*}
                C_{XY}(f)&=\frac{\left|S_{XY}(f)\right|^2}{S_{XX}(f)S_{YY}(f)}
            \end{align*}

    \end{itemize}

\end{frame}

\begin{frame}
    \frametitle{Multi-trial spectral coherence: intuition}

    \begin{figure}
    \begin{center}

        \includegraphics[width=3in]{figures/spectralCoherenceIntuition.pdf}
        \caption{Spectral coherence measures constant phase difference between
        two times series at a given frequency.}
    \end{center}
    \end{figure}

\end{frame}

\begin{frame}
    \frametitle{Multi-trial spectral coherence: interpretation}

    \begin{itemize}

        \item define

            \begin{align*}
                X(f) & = A(f)e^{j\phi_X(f)}\\
                Y(f) & = B(f)e^{j\phi_Y(f)}
            \end{align*}

        \item then

            \begin{align*}
                S_{XY,k}(f)&=X_kY^*_k\\
                           &=A_ke^{j\phi_{xk}}\left(B_ke^{j\phi_{yk}}\right)^*\\
                           &=A_ke^{j\phi_{xk}}\left(B_ke^{-j\phi_{yk}}\right)\\
                           &=A_kB_ke^{j(\phi_{xk}-\phi_{yk})}\\
            \end{align*}

    \end{itemize}

\end{frame}

\begin{frame}
    \frametitle{Multi-trial spectral coherence: interpretation}

    \begin{center}
        \begin{figure}
            \includegraphics[width=4in]{figures/crossPowerGraphical.pdf}
        \end{figure}
    \end{center}

\end{frame}

\begin{frame}
    \frametitle{Multi-trial spectral coherence: interpretation}

    Given its definition, \textbf{multi-trial spectral coherence}

    \begin{align*}
        C_{XY}(f)&=\frac{\left|<S_{XY,k}(f)>\right|^2}{<S_{XX}(f)><S_{YY}(f)>}
    \end{align*}

    corresponds to averaging the cross-power vectors and normalizing the result
    by the corresponding power spectrum terms.

    \begin{center}
        \begin{figure}
            \includegraphics[width=3in]{figures/averagingCrossSpectrums.pdf}
        \end{figure}
    \end{center}

\end{frame}

\begin{frame}
    \frametitle{Multi-trial spectral coherence: interpretation}

    \textcolor{red}{Multi-trial spectral coherence at frequency $f$ is large
    when the phase difference at frequency $f$ is approximately constant across
    trials.}

\end{frame}

\section{Understanding the plots we will generate in the next worksheet}

\begin{frame}
    \frametitle{Understanding the plots we will generate in the next worksheet}

    Please refer to the plots in the
    \href{https://drive.google.com/file/d/1l9bymhvPO8CdKJZgVTbJih2JIliLG9mm/view}{next worksheet}.

\end{frame}

\begin{frame}
    \frametitle{Summary}

\end{frame}

\begin{frame}
    \frametitle{Bibliography}

    \bibliographystyle{apalike}
%     \bibliography{communications,signalProcessing}

\end{frame}

\end{document}
