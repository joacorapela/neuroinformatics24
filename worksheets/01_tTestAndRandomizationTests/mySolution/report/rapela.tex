
\documentclass[12pt]{article}

\usepackage{graphicx}
\usepackage[shortlabels]{enumitem}
\usepackage[colorlinks=]{hyperref}
\usepackage[margin=2cm]{geometry}
\usepackage{amsmath}
\usepackage{listings,lstautogobble}

\lstset{language=Java,
    autogobble=true
 }

\title{Worksheet: hypothesis tests}
\author{Joaquin Rapela}

\begin{document}

\maketitle

\begin{enumerate}
    \item \textbf{Detailed hypothesis test for example 2 in discussion notes}

        \begin{description}

            \item [Identify the null hypothesis $\mathcal{H}_0$:] the mean peak
                visual ERP in medicated subjects is 2~mV

            \item [Identify the alternative hypothesis $\mathcal{H}_a$:] the
                mean peak visual ERP in medicated subjects is different from
                2~mV.

            \item [Select a test statistic:] standarized sample mean $Z$.

            \item [Calculate the observed value of the test statistic:]
                $z=\frac{\bar{x}-\mu_0}{s/\sqrt{n}}=\frac{1.3-2}{2.6/\sqrt{50}}=-1.9$

            \item [Calculate the p-value:]
                $\text{p\_value}=P(-|z|>Z)+P(Z>|z|)=2 P(Z>|z|)=0.057$.

            \item [Draw our conclusion:] $\text{p\_value}=0.057>0.05$ then
                \textbf{do not reject $\mathcal{H}_0$}.

        \end{description}

        A Python script to solve this exercise can be found
        \href{https://github.com/joacorapela/neuroinformatics24/blob/master/worksheets/01_tTestAndRandomizationTests/mySolution/code/doZTest.py}{here}
        and a shell script with the corresponding parameters can be found
        \href{https://github.com/joacorapela/neuroinformatics24/blob/master/worksheets/01_tTestAndRandomizationTests/mySolution/code/doEx1.csh}{here}.

    \item 
        \begin{enumerate}[(a)]
            \item
                \begin{description}
                    \item[$\mathcal{H}_0$:] the population mean is $\mu_0=2.3$
                    \item[$\mathcal{H}_a$:] the population mean is $\mu_0>2.3$
                \end{description}

            \item Because $n>30$ it is reasonable to assume that
                $Z\sim\mathcal{N}(0,1)$. Then the rejection region is
                $z>z_\alpha$, with $\alpha=0.05$.

            \item Roughly, for $Y\sim\mathcal{N}(\mu,\sigma^2)$ there is a considerable
                probability of obtaining a sample in the range $[\mu,
                \mu+2\sigma]$. Because under the null hypothesis
                $\bar{X}\sim\mathcal{N}(\mu_0,s/\sqrt{n})$, there is a
                considerable probabability of obtaining a sample of $\bar{X}$
                in the range $[\mu_0,\mu_0+2s/\sqrt{n}]$.

                $s/\sqrt{n}\sim 0.3/6\sim 0.05$. Thus, there is a
                considerable probability of obtaining by chance a sample of
                $\bar{X}$ in the range $[2.3, 2.3+2\ 0.05]=[2.3,2.4]$. Because
                2.4 is in the boundary of this interval, it is not obvious if
                a hypothesis test will reject or not the null hypothesis.

                Lets do the test. We first compute the observed test statistic:

                \begin{align*}
                    z&=\frac{\bar{x}-\mu_0}{s/\sqrt{n}}=\frac{2.4-2.3}{0.29/\sqrt{35}}=2.04
                \end{align*}

                The p-value corresponding to this observed statistic is
                $p=0.02$, so we reject the
                null hypothesis with a confidence level $\alpha=0.05$.

        \end{enumerate}

        A Python script to solve this exercise can be found
        \href{https://github.com/joacorapela/neuroinformatics24/blob/master/worksheets/01_tTestAndRandomizationTests/mySolution/code/doZTest.py}{here}
        and a shell script with the corresponding parameters can be found
        \href{https://github.com/joacorapela/neuroinformatics24/blob/master/worksheets/01_tTestAndRandomizationTests/mySolution/code/doEx2.csh}{here}.

    \item \textbf{Potency of an antibiotic}
        \begin{enumerate}[(a)]
            \item $\mathcal{H}_0$: the mean potency of the antibiotic is $\mu_0=80\%$.
            \item $\mathcal{H}_a$: the mean potency of the antibiotic is $\mu_0\ne 80\%$.
            \item because $n=100$ it is sensible to  assume
                $Z\sim\mathcal{N}(0,1)$. I will perform a two-tailed z-test
                with $\bar{x}=79.7\%, \mu_0=80.0\%, n=100, s=0.8$ and
                $\alpha=.05$. Lets compute the observed statistic.

                \begin{align*}
                    z &= \frac{\bar{x}-\mu_0}{s/\sqrt{n}}=\frac{29.7-80}{0.8/\sqrt{100}}=-3.75
                \end{align*}

                The p-value corresponding to this observed statistic is
                $p<0.0002$, so we reject the
                null hypothesis with a confidence level $\alpha=0.05$.

        \end{enumerate}

        A Python script to solve this exercise can be found
        \href{https://github.com/joacorapela/neuroinformatics24/blob/master/worksheets/01_tTestAndRandomizationTests/mySolution/code/doZTest.py}{here}
        and a shell script with the corresponding parameters can be found
        \href{https://github.com/joacorapela/neuroinformatics24/blob/master/worksheets/01_tTestAndRandomizationTests/mySolution/code/doEx3.csh}{here}.

    \item \textbf{Smoking and lung capacity}
        Because $n=20$ it is not safe to assume $\bar{Z}~\mathcal{N}(0,1)$. We
        will perform a right-tailed t-test instead. Lets compute the observed
        statistic.

        \begin{align*}
            t &= \frac{\bar{x}-\mu_0}{s/\sqrt{n}}=\frac{89.85-100}{14.53/\sqrt{20}}=-3.12
        \end{align*}

        The p-value corresponding to this observed statistic is
        $p<0.003$, so we reject the
        null hypothesis with a confidence level $\alpha=0.01$.

        A Python script to solve this exercise can be found
        \href{https://github.com/joacorapela/neuroinformatics24/blob/master/worksheets/01_tTestAndRandomizationTests/mySolution/code/doTTest.py}{here}
        and a shell script with the corresponding parameters can be found
        \href{https://github.com/joacorapela/neuroinformatics24/blob/master/worksheets/01_tTestAndRandomizationTests/mySolution/code/doEx4.csh}{here}.

    \item \textbf{Power of a test}

        The power of a statistical test is $1-\beta$, where $\beta$ is the
        probability of type II error (i.e., the probability that the null
        hypothesis is not rejected given that an alternative hypothesis is
        true).

        To calculate $\beta$ for a right-tailed test, I first find the critical value of the sample
        mean, $\bar{x}_c$, such that
        $P(\bar{X}>\bar{x}_c|\mathcal{H}_0)=\alpha$ (note that for a
        right-tailed test
        $\alpha=P(Z>z_\alpha|\mathcal{H}_0)=P(\frac{\bar{X}-\mu_0}{s/\sqrt{n}}>z_\alpha|\mathcal{H}_0)=P(\bar{X}>\mu_0+z_\alpha\ s/\sqrt{n}|\mathcal{H}_0)$, so that $\bar{x}_c=\mu_0+z_\alpha\ s/\sqrt{n}$).
        Then
        $\beta=P(\bar{X}<\bar{x}_c|\mathcal{H}_a)=P(\frac{\bar{X}-\mu_a}{s/\sqrt{n}}<\frac{\bar{x}_c-\mu_a}{s/\sqrt{n}}|\mathcal{H}_a)=P(Z<\frac{\bar{x}_c-\mu_a}{s/\sqrt{n}})=\Phi(\frac{\bar{x}_c-\mu_a}{s/\sqrt{n}})$,
        where $\Phi(x)=P(Z<x)$ is the standard normal cummulative distribution
        function. The power of the test
        is $1-\beta$. $\beta$ is the the red area under in Figure~2 of the
        \href{https://github.com/joacorapela/neuroinformatics24/blob/master/practicals/01_tTestAndRandomizationTests/introAndHipothesisTests.pdf}{discusion
        slides}. Code to compute $\beta$ is given in the listing below.

        \begin{lstlisting}[language=Python]
            def compute_beta(mu_Ha, critical_value, ste_mean):
                beta = scipy.stats.norm.cdf((critical_value - mu_Ha) / ste_mean)
                return beta
        \end{lstlisting}

        \begin{enumerate}
            \item the calculated powers are:
                \begin{description}
                    \item[$\mu_\text{Ha}=108$:] power=0.37
                    \item[$\mu_\text{Ha}=112$:] power=0.75
                    \item[$\mu_\text{Ha}=116$:] power=0.95
                \end{description}

                Power versus effect size is shown in
                Figure~\ref{fig:powerVsEffectSize}.

                \begin{figure}
                    \begin{center}
                        \includegraphics[width=4in]{../figures/powerVsEffectSize.png}
                    \end{center}
                    \caption{Power versus effect size.}
                    \label{fig:powerVsEffectSize}
                \end{figure}

                A Python script solving this item appears
                \href{https://github.com/joacorapela/neuroinformatics24/blob/master/worksheets/01_tTestAndRandomizationTests/mySolution/code/doPowerExa.py}{here}.

            \item Figure~\ref{fig:powerAtSignificanceLevels} shows power plots for
                $\alpha=0.01$ and $\alpha=0.05$.

                \begin{figure}
                    \begin{center}
                        \includegraphics[width=4in]{../figures/powerAtSignificanceLevels.png}
                    \end{center}
                    \caption{Power plots for significance levels $\alpha=0.01$ and
                    $\alpha=0.05$.}
                    \label{fig:powerAtSignificanceLevels}
                \end{figure}

                A Python script solving this item appears
                \href{https://github.com/joacorapela/neuroinformatics24/blob/master/worksheets/01_tTestAndRandomizationTests/mySolution/code/doPowerExb.py}{here}.

            \item Figure~\ref{fig:powerAtNs} shows power plots for
                $n=16$ and $n=65$.

                \begin{figure}
                    \begin{center}
                        \includegraphics[width=4in]{../figures/powerAtNs.png}
                    \end{center}
                    \caption{Power plots for $n=16$ and $n=65$.}
                    \label{fig:powerAtNs}
                \end{figure}

                A Python script solving this item appears
                \href{https://github.com/joacorapela/neuroinformatics24/blob/master/worksheets/01_tTestAndRandomizationTests/mySolution/code/doPowerExc.py}{here}.

            \item following the exercise hints I first derived an expression
                for the critical value of the sample mean only considering the
                type I error, as we did above.

                \begin{align}
                    \alpha=P(Z>z_\alpha|\mathcal{H}_0)&=P(\frac{\bar{X}-\mu_0}{s/\sqrt{n}}>z_\alpha|\mathcal{H}_0)=P(\bar{X}>\mu_0+z_\alpha\
                    s/\sqrt{n}|\mathcal{H}_0)\nonumber\\
                    \text{then}&\quad\bar{x}_{c0}=\mu_0+z_\alpha\ s/\sqrt{n}\label{eq:nullCritical}
                \end{align}

                then I derived another expression for the critical value of the
                sample mean only considering the type II error

                \begin{align}
                    \beta=P(Z<-z_\beta|\mathcal{H}_a)&=P(\frac{\bar{X}-\mu_a}{s/\sqrt{n}}<-z_\beta|\mathcal{H}_a)=P(\bar{X}<\mu_a-z_\beta\
                    s/\sqrt{n}|\mathcal{H}_a)\nonumber\\
                    \text{then}&\quad\bar{x}_{ca}=\mu_a-z_\beta\ s/\sqrt{n}\label{eq:altCritical}
                \end{align}

                Finally I equate $\bar{x}_{c0}$ in Eq.~\ref{eq:nullCritical}
                with $\bar{x}_{ca}$ in Eq.~\ref{eq:altCritical} and solve for
                $n$

                \begin{align*}
                    \bar{x}_{c0}&=\bar{x}_{ca}\quad\text{iff}\\
                    \mu_0+z_\alpha\ s/\sqrt{n}&=\mu_a-z_\beta\ s/\sqrt{n}\quad\text{iff}\\
                    n&=\left(s\frac{z_\beta+z_\alpha}{\mu_a-\mu_0}\right)^2
                \end{align*}

                the minimum sample size to achieve significance level of
                $\alpha=0.01$ and power of $1-\beta=0.8$ for $\mathcal{H}_a:
                \mu_a=112$ is $n=18$. Figure~\ref{fig:powerPlotForMinN} shows
                the power plot for this sample size. As required for
                $\mu_a=112$ the power of the test is $1-\beta=0.8$.

                A Python script solving this item appears
                \href{https://github.com/joacorapela/neuroinformatics24/blob/master/worksheets/01_tTestAndRandomizationTests/mySolution/code/doPowerExd.py}{here}.

                \begin{figure}
                    \begin{center}
                        \includegraphics[width=4in]{../figures/powerForMinN_muHa112.00.png}
                    \end{center}
                    \caption{Power plots for $n=11$, $\alpha=0.05$, and
                    $\beta=0.2$. As required for $\mu_a=112$ the power of the
                    test is $1-\beta=0.8$.}
                    \label{fig:powerPlotForMinN}
                \end{figure}


        \end{enumerate}
\end{enumerate}

\end{document}
