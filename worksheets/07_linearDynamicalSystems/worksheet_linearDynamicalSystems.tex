\documentclass[12pt]{article}

\usepackage{natbib}
\usepackage{apalike}
\usepackage{listings}
\usepackage{xcolor}
\usepackage{graphicx}

\usepackage[shortlabels]{enumitem}
\usepackage[colorlinks=]{hyperref}
\usepackage[margin=2cm]{geometry}

\definecolor{codegreen}{rgb}{0,0.6,0}
\definecolor{codegray}{rgb}{0.5,0.5,0.5}
\definecolor{codepurple}{rgb}{0.58,0,0.82}
\definecolor{backcolour}{rgb}{0.95,0.95,0.92}

\lstdefinestyle{mystyle}{
    backgroundcolor=\color{backcolour},
    commentstyle=\color{codegreen},
    keywordstyle=\color{magenta},
    numberstyle=\tiny\color{codegray},
    stringstyle=\color{codepurple},
    basicstyle=\ttfamily\footnotesize,
    breakatwhitespace=false,
    breaklines=true,
    captionpos=b,
    keepspaces=true,
    numbers=left,
    numbersep=5pt,
    showspaces=false,
    showstringspaces=false,
    showtabs=false,
    tabsize=2
}

\lstset{style=mystyle}

\title{Worksheet: Linear Dynamical Systems}
\author{Joaquin Rapela and Aniruddh Galgali}

\begin{document}

\maketitle

\section{Sampling from a linear dynamical system}

Sample (i.e., simulate) observations from a linear dynamical system for
tracking.  Follow the equations on slide 16 of the
\href{https://github.com/joacorapela/neuroinformatics24/blob/master/lectures/07_linearDynamicalSystems/LDS_SWCNeuroinf2024.pdf}{Linear
Dynamical Systems} lecture. That is, first sample an initial state
$\mathbf{x}_1$, then sample the remaining states
$\mathbf{x}_2,\ldots,\mathbf{x}_N$, and finally simple the observations
$\mathbf{y}_1,\ldots,\mathbf{y}_N$.

You may use the script
\href{https://github.com/joacorapela/neuroinformatics24/blob/master/worksheets/07_linearDynamicalSystems/code/scripts/doSimulateTrajectoryDWPA.py}{doSimulateTrajectoryDWPA.py}
in the class repository, but you will need to complete a few missing parts in
the module
\href{https://github.com/joacorapela/neuroinformatics24/blob/master/worksheets/07_linearDynamicalSystems/code/src/simulation.py}{simulation.py}.
You should obtain a plot similar to that in Figure~\ref{fig:simulated_pos}.

\begin{figure}
	\begin{center}
		\includegraphics[width=5in]{figures/simulated_pos.png}
		\label{fig:simulated_pos}
		\caption{Simulated observations and state position components
              using the Discrete Wiener Process Acceleration
              model~\citep[][Section 6.3.3]{barShalomEtAl04}.}
	\end{center}
\end{figure}

\section{Tracking a foraging mouse}

Here we are going to track a foraging mouse using linear dynamical
systems\footnote{The video for this example was generously provided by the
Sainsbury Wellcome Centre Foraging Behaviour Working Group (2023). Aeon: An
open-source platform to study the neural basis of ethological behaviours over
naturalistic timescales, \url{https://doi.org/10.5281/zenodo.8413142}.}. For
this you may want to use the script
\href{https://github.com/joacorapela/neuroinformatics24/blob/master/worksheets/07_linearDynamicalSystems/code/scripts/doTrackMouse.py}{doTrackMouse.py}.
This script reads a video file and using the \href{https://opencv.org/}{opencv}
computer vision library it estimates the filtering distribution of the center of
mass of the foraging mouse. It plots a red point in the video frame at the
position of the opencv estimate.

This opencv estimate is noisy and sometimes is not available due to occlusions.
Here we use the Kalman filter to estimate the true center of mass of the mouse
from the noisy observations. At each time point the Kalman filter provides the
mean and covariance of the posterior distribution of the center of mass. The
script plots a greeen point in the video frame at the location of this mean and
a 95\% confidence ellipse summarizing the variability of the posterior
distribution. Figure~\ref{fig:trackedMouse} plots one example frame of the
video generated by the script.

\begin{figure}
	\begin{center}
		\includegraphics[width=5in]{figures/trackedMouse.png}
		\label{fig:trackedMouse}

        \caption{Example frame of the video generated by the
        \href{https://github.com/joacorapela/neuroinformatics24/blob/master/worksheets/07_linearDynamicalSystems/code/scripts/doTrackMouse.py}{doTrackMouse.py}
        script. The opencv estimate of the mouse position is plotted in red,
        the Kalman filter estimateed mean of the position is plotted in green,
        and the 95\% confidence ellipse of the estimated position is drawn in
        green.}

	\end{center}
\end{figure}


\bibliographystyle{plainnat}
\bibliography{tracking}

\end{document}
