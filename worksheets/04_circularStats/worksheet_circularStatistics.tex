\documentclass[12pt]{article}

\usepackage{graphicx}
\usepackage{natbib}
\usepackage{apalike}
\usepackage{listings}
\usepackage{amsmath}
\usepackage{xcolor}
\usepackage{subcaption}

\usepackage[shortlabels]{enumitem}
\usepackage[colorlinks]{hyperref}
\usepackage[margin=2cm]{geometry}

\definecolor{codegreen}{rgb}{0,0.6,0}
\definecolor{codegray}{rgb}{0.5,0.5,0.5}
\definecolor{codepurple}{rgb}{0.58,0,0.82}
\definecolor{backcolour}{rgb}{0.95,0.95,0.92}

\lstdefinestyle{mystyle}{
    backgroundcolor=\color{backcolour},   
    commentstyle=\color{codegreen},
    keywordstyle=\color{magenta},
    numberstyle=\tiny\color{codegray},
    stringstyle=\color{codepurple},
    basicstyle=\ttfamily\footnotesize,
    breakatwhitespace=false,         
    breaklines=true,                 
    captionpos=b,                    
    keepspaces=true,                 
    numbers=left,                    
    numbersep=5pt,                  
    showspaces=false,                
    showstringspaces=false,
    showtabs=false,                  
    tabsize=2
}

\lstset{style=mystyle}

\title{Worksheet: Circular statistics}
\author{Joaquin Rapela}

\begin{document}

\maketitle

In this worksheets you will practice:

\begin{itemize}

    \item accessing electrophysiology data from
        \href{https://dandiarchive.org/}{Dandi},

    \item bandpass filtering a signal,

    \item calculating the instantaneous amplitude and phase of a signal with
        the Hilbert transform,

    \item computing the circular mean of a set of phases,

    \item testing for non-uniformity of a set of circular variables with the
        Rayleigh test,

    \item detecting travelling waves in local field potentials,

    \item performing linear regression analysis.

\end{itemize}

You will quantitatively characterise travelling waves in electro-corticographic
recording from humans during the production of consonant vowel syllables, as
described in
\citet{rapelaInPrepTWsInSpeech,rapelaInPrepSyncTWs,rapelaInPrepSyncTWsII}. A
video illustrating these travelling waves can be found
\href{https://www.youtube.com/watch?v=6QYUGRqZ7Hc}{here}. This video shows the
local field potential (LFP) voltages bandpass filtered between 0.4 and 0.8~Hz,
around the mean frequency of consonant-vowel syllable production of 0.62~Hz.

\section{Install the Python packages required to obtain data from Dandi}

\begin{verbatim}
# functionality to manipulate data in the NWB format
conda install pynwb

# functionality to access data from Dandi
pip install dandi

# this package contains the function rayleightest
pip install astropy

\end{verbatim}

\section{Download ECoG recordings from Dandi}

The script
\href{https://github.com/joacorapela/neuroinformatics24/blob/master/worksheets/04_circularStats/code/scripts/doDownloadData.py}{doDownloadData.py}
illustrates how to use the Dandi Python API to download an ephys data set. The
default parameters of this script are set to download ECoG recordings from the
subject and session analysed in
\citet{rapelaInPrepTWsInSpeech,rapelaInPrepSyncTWs,rapelaInPrepSyncTWsII}.

This script saves a Numpy \texttt{.npz} file containing the items:

\begin{description}

    \item[\texttt{voltages}]: array of size \texttt{n\_electrodes x n\_samples}
        such that \texttt{voltages[i,j]} is the recorded voltage at electrode
        $i$ and sample point $j$.

    \item[\texttt{srate}]: recording sample rate (Hz).

    \item[\texttt{electrodes}]: electrode numbers saved. The default parameters
        of the script only save electrodes 135 to 142.

    \item[\texttt{cvs\_transition\_times}]: transition times between consonant
        and vowels (i.e., \texttt{cvs\_transition\_times[i]} is the time at
        which the subject transitioned between the consonant and the vowel of
        the ith syllable).

\end{description}

\section{Calculate the mean frequency of consonant-vowel syllable production}

Plot a histogram of times between production of consecutive consonant vowel
syllables. Calculate the median of these times and add it to the title of the
histogram. Refer to Fig.~\ref{fig:isis}.

We called the inverse of this number the mean consonant vowel
production frequency. Below we narrow filter the ECoG recordings around this
frequency.

\begin{figure}
    \begin{center}
        \includegraphics[width=4in]{figures/000019_sub-EC2_ses-EC2-B105_\[135,136,137,138,139,140,141,142\]_ISI.png}
    \end{center}
    \caption{Histogram of inter-syllable intervals}
    \label{fig:isis}
\end{figure}

\section{Bandpass filter the raw voltages}

The script
\href{https://github.com/joacorapela/neuroinformatics24/blob/master/worksheets/04_circularStats/code/scripts/doBandpassVoltages.py}{doBandpassVoltages.py}
bandpasses the recorded Voltages between 0.4 and 0.8~Hz. It uses a second order
butterworth filter (function \texttt{scipy.signal.butter}) and forward-backward
filtering (function \texttt{scipy.signal.filtfilt}) to not introduce phase
delays. It saves the filtered voltages as variable \texttt{filtered\_voltages}.

Use the filtered voltages to reproduce Figure~6 from
\citet{rapelaInPrepTWsInSpeech} with electrodes 135-142
(Figure~\ref{fig:filteredVoltages}). Can you see the phase delays between the
different electrode waveforms?

\begin{figure}
    \begin{center}
        \includegraphics[width=4in]{figures/000019_sub-EC2_ses-EC2-B105_\[135,136,137,138,139,140,141,142\]_0.4-0.8Hz_voltages.png}
    \end{center}
    \caption{Voltages filtered between 0.4 and 0.8~Hz}
    \label{fig:filteredVoltages}
\end{figure}

\section{Compute the Hilbert transform of the filtered voltages}

The script
\href{https://github.com/joacorapela/neuroinformatics24/blob/master/worksheets/04_circularStats/code/scripts/doHilbertTransform.py}{doHilbertTransform.py}
calculates the Hilbert transform of the bandpass filtered voltages. It saves
the complex time series output of the Hilbert transform in variable
\texttt{ht\_filtered\_voltages}. The phase of this time
series at time $t$ is the instantaneous phase of the original voltages at time
$t$ in the filtered filtered frequency range 0.4-0.8~Hz. We will use these
phases below.

\section{Plot phase histograms, compute the mean resultant vector and test for circular non-uniformity}

Reproduce Figure~17 from \citet{rapelaInPrepSyncTWs}.

First extract the phases at the times of consonant-vowel transitions. The
following code snippet illustrates one way of doing this.

\begin{lstlisting}[language=python]
load_res = np.load(ht_filename)
ht_filtered_voltages = load_res["ht_filtered_voltages"]
srate = load_res["srate"]
electrodes = load_res["electrodes"]
cvs_transition_times = load_res["cvs_transition_times"]

times = np.arange(0, ht_filtered_voltages.shape[1]) / srate
cvs_trans_samples = [np.argmin(np.abs(times-cvs_transition_time))
                     for cvs_transition_time in cvs_transition_times]

phases = np.angle(ht_filtered_voltages)
cvs_transition_phases = phases[:, cvs_trans_samples]
\end{lstlisting}

Then for each electrode separately calculate the mean resultant vector

\begin{align}
    \bar{\mathbf{R}}=\frac{1}{N}\sum_{i=1}^Nvec(\theta_i)
\end{align}

The following function does this

\begin{lstlisting}[language=python]

def calculateMeanResultantVector(angles):
    vectors = np.array([np.exp(1j*angle) for angle in angles])
    mean_resultant_vector = vectors.mean()
    return mean_resultant_vector

\end{lstlisting}

\noindent where \texttt{angles} represent the phases of from a given electrode.

Next plot a circular histogram of the phases of an electrode. Add to this
histogram a vector with radius equal to the absolute value of the mean
resultant vector and angle equal to the phase of this vector.

Add to the title of this plot the p-value of a Rayleigh non-uniformity test. To compute this p-value use the function
\texttt{astropy.stats.rayleightest} (Figure~\ref{fig:phasesHist}).

\begin{figure}
	\centering
	\begin{subfigure}{0.4\textwidth}
        \includegraphics[width=\textwidth]{figures/000019_sub-EC2_ses-EC2-B105_\[135,136,137,138,139,140,141,142\]_0.4-0.8Hz_HT_phasesHist_elect135.png}
    	\caption{Electrode 135.}
	\end{subfigure}
	\hfill
	\begin{subfigure}{0.4\textwidth}
        \includegraphics[width=\textwidth]{figures/000019_sub-EC2_ses-EC2-B105_\[135,136,137,138,139,140,141,142\]_0.4-0.8Hz_HT_phasesHist_elect136.png}
    	\caption{Electrode 136.}
	\end{subfigure}
	\hfill
	\begin{subfigure}{0.4\textwidth}
        \includegraphics[width=\textwidth]{figures/000019_sub-EC2_ses-EC2-B105_\[135,136,137,138,139,140,141,142\]_0.4-0.8Hz_HT_phasesHist_elect137.png}
    	\caption{Electrode 137.}
	\end{subfigure}
	\hfill
	\begin{subfigure}{0.4\textwidth}
        \includegraphics[width=\textwidth]{figures/000019_sub-EC2_ses-EC2-B105_\[135,136,137,138,139,140,141,142\]_0.4-0.8Hz_HT_phasesHist_elect138.png}
    	\caption{Electrode 138.}
	\end{subfigure}
	\caption{Histograms of phases at the times of consonant-vowel transitions at different electrodes along the dorso-ventral axis.}
    \label{fig:phasesHist}
\end{figure}

\section{Checking for travelling waves events}

Reproduce Figure~15 from \citet{rapelaInPrepSyncTWsII}.

Select a time of your interest to check if at this time a travelling wave was propagating among electrodes 135-142.

First compute the phase differences of all electrodes with respect to electrode 142. Because phases, and phase differences, are invariant to the addition of multiples of $2\pi$, we will unwrap the phases differences so that they all are in a consistent range, using the function \texttt{np.unwrap} that this. The following code does this.

\begin{lstlisting}[language=python]
    phases = np.angle(ht_filtered_voltages[:, wave_event_sample])
    phase_diffs = np.unwrap(phases - phases[-1])
\end{lstlisting}

Then calculate the distances between the different electrodes and electrode 142, using an inter-electrode distance of 4~mm. Finally, perform a linear regression analysis between the electrodes phase differences and electrodes separation with respect to electrode 142.

\begin{lstlisting}[language=python]
electrodes_distances = np.arange(len(electrodes)-1, 0-1, -1) * electrodes_separation
lm_res = scipy.stats.linregress(x=electrodes_distances, y=phase_diffs)
\end{lstlisting}

Draw a scatter plot of phase differences versus electrodes separations and superimpose the best fitting line. Add to the the title the slope of this line (i.e., the speed of the travelling wave) and the p-value of the linear regression analysis (Figure~\ref{fig:waveEvent}).

\begin{figure}
    \begin{center}
        \includegraphics[width=4in]{figures/000019_sub-EC2_ses-EC2-B105_[135,136,137,138,139,140,141,142]_0.4-0.8Hz_HT_waveEvent115.7615608.png}
    \end{center}
    \caption{Wave event among electrodes 135 and 142 at time 115.76~secs.}
    \label{fig:waveEvent}
\end{figure}

% \bibliographystyle{apsr}
\bibliographystyle{plainnat}
\bibliography{travelingWaves}

\end{document}
